\documentclass[12pt]{article}
\usepackage[utf8]{inputenc} % Para caracteres en español
\usepackage{amsmath,amsthm,amsfonts,amssymb,amscd}
\usepackage{multirow,booktabs}
\usepackage[table]{xcolor}
\usepackage{fullpage}
\usepackage{tikz}
\usepackage{physics}

%%
\usepackage{pgfplots}
\usepgfplotslibrary{polar}
\usepgflibrary{shapes.geometric}
\usetikzlibrary{calc}


\pgfplotsset{my style/.append style={axis x line=middle, axis y line=
           middle, xlabel={$x$}, ylabel={$y$}, axis equal }}
\pgfplotsset{my style-2/.append style={axis x line=middle, axis y line=
           middle, xlabel={$x$}, ylabel={$y$}}}


\usetikzlibrary{arrows, circuits.ee.IEC, positioning}
\usepackage[american voltages, american currents,siunitx]{circuitikz}

\usepackage{lastpage}
\usepackage{enumitem}
\usepackage{fancyhdr}
\usepackage{mathrsfs}
\usepackage{wrapfig}
\usepackage{setspace}
\usepackage{calc}
\usepackage{multicol}
\usepackage{cancel}
\usepackage[retainorgcmds]{IEEEtrantools}
\usepackage[margin=3cm]{geometry}
\usepackage{amsmath}
\newlength{\tabcont}
\setlength{\parindent}{0.0in}
\setlength{\parskip}{0.05in}
\usepackage{empheq}
\usepackage{framed}
\usepackage[most]{tcolorbox}
\usepackage{xcolor}
\colorlet{shadecolor}{orange!15}
\parindent 0in
\parskip 12pt
\geometry{margin=1in, headsep=0.25in}
\theoremstyle{definition}
\newtheorem{defn}{Definition}
\newtheorem{reg}{Rule}
\newtheorem{exer}{Exercise}
\newtheorem{note}{Note}
\usepackage[english]{babel}
\usepackage[protrusion=true,expansion=true]{microtype}  
\usepackage{amsmath,amsfonts,amsthm,amssymb}
\usepackage{graphicx}
\usepackage{tcolorbox}
\usepackage{amssymb}
\tcbuselibrary{theorems}
\newtcbtheorem
  []% init options
  {definition}% name
  {Definition}% title
  {%
    colback=blue!1,
    colframe=blue!75!black,
    fonttitle=\bfseries,
  }% options
  {def}% prefix

\newtcbtheorem
  []% init options
  {lemma}% name
  {Lemma}% title
  {%
    colback=yellow!5,
    colframe=orange!110,
    fonttitle=\bfseries,
  }% options
  {lem}% prefix

\newtcbtheorem
  []% init options
  {theorem}% name
  {Theorem}% title
  {%
    colback=yellow!5,
    colframe=orange!110,
    fonttitle=\bfseries,
  }% options
  {thm}% prefix

\newtcbtheorem
  []% init options
  {solution}% name
  {Solution}% title
  {%
    colback=magenta!5,
    colframe=magenta!150,
    fonttitle=\bfseries,
  }% options
  {not}% prefix

\newtcbtheorem
  []% init options
  {example}% name
  {Example}% title
  {%
    colback=red!5,
    colframe=red!100,
    fonttitle=\bfseries,
  }% options
  {exm}% prefix


\newtcbtheorem
  []% init options
  {algorithm}% name
  {Algorithm} %title
  {%
    colback=blue!5,
    colframe=blue!100,
    fonttitle=\bfseries,
  }% options
  {alg}% prefix

\newtcbtheorem
  []% init options
  {properties}% name
  {Properties}% title
  {%
    colback=green!5,
    colframe=green!35!black,
    fonttitle=\bfseries,
  }% options
  {pro}% prefix



% Sets
\newcommand{\N}{\ensuremath{\mathbb{N}}}
\newcommand{\Z}{\ensuremath{\mathbb{Z}}}
\newcommand{\Q}{\ensuremath{\mathbb{Q}}}
\newcommand{\R}{\ensuremath{\mathbb{R}}}
\newcommand{\C}{\ensuremath{\mathbb{C}}}
\newcommand{\F}{\ensuremath{\mathbb{F}}}
\newcommand{\Mnn}{\ensuremath{M_{n \times n}}}
\newcommand{\sym}{\mathbin{\triangle}}
\newcommand{\stcomp}[1]{{#1}^\complement}

% Operators
\newcommand{\Rarr}{\Rightarrow}
\newcommand{\Larr}{\Leftarrow}
\newcommand{\Harr}{\Leftrightarrow}
\newcommand{\harr}{\leftrightarrow}
\newcommand{\dyx}{\dv{y}{x}}

% Linear Algebra
\newcommand{\vui}{\mathbf{\hat{\textnormal{\bfseries\i}}}}
\newcommand{\vuj}{\mathbf{\hat{\textnormal{\bfseries\j}}}}
\newcommand{\xto}{\xrightarrow} % \xto{R_1 \harr R_2}
\newcommand{\cbm}[2]{\ensuremath{{}_{#1}[I]{}_{#2}}} % change of basis matrix
\DeclareMathOperator{\Proj}{Proj}
\DeclareMathOperator{\Perp}{Perp}
\DeclareMathOperator{\Span}{Span}
\DeclareMathOperator{\Col}{Col}
\DeclareMathOperator{\adj}{adj}

\renewcommand\qedsymbol{$\blacksquare$}

\newcommand{\HRule}[1]{\rule{\linewidth}{#1}}   % Horizontal rule

\makeatletter             % Title
\def\printtitle{%           
    {\centering \@title\par}}
\makeatother                  

\makeatletter             % Author
\def\printauthor{%          
    {\centering \large \@author}}       
\makeatother          
\makeatother  
\title{ \normalsize \textsc{}   % Subtitle
      \\[2.0cm]               % 2cm spacing
      \HRule{0.5pt} \\            % Upper rule
      \LARGE \textbf{\uppercase{Math 116 Handbook}} % Title
      \HRule{2pt} \\ [0.5cm]    % Lower rule + 0.5cm spacing
      \normalsize     % Todays date
    }

\author{
    Abdullah Zubair\\
        \texttt{a6zubair@uwaterloo.ca} \\
}



\begin{document}
\setcounter{section}{2}
\section{Linear Independence, Span, Basis, Dimension}
\subsection{Linear Independence}
\begin{definition}{Linear Independence}{label}
    Let $V$ be a vector space and let $S\subseteq V$ (Possibly infinite). We say that $S$ is \emph{linearly independent} if the only solution to a finite linear combination of vectors that result in the zero vector is when all coefficients are zero. Or,
    $$\forall v_1,\dots,v_n\in S, c_i\in \F, c_1v_1 + \dots + c_nv_n = 0 \implies c_1 = c_2 = \dots = c_n = 0$$

\end{definition}


\subsection{The Span of a Set}
\begin{definition}{The Subspace Spanned by a Set}{label}
Let $V$ be a vector space over $\F$, and let $S$ be a subset of $V$. The subspace spanned by $S$, $\Span S$,
is the smallest subspace of $V$ containing $S$ (Set containment). $\Span S$ is determined by the following
properties:
\begin{enumerate}
    \item $\Span S$ is a subspace of $V$, and $S\subseteq \Span S$.
    \item If $W$ is a subspace of $V$ such that $S\subseteq W$, then $\Span S \subseteq W$.
\end{enumerate}
\end{definition}


\begin{theorem}{}{label}
Let $V$ be a vector space, and let $S$ be a subset of $V$. Then:
\begin{enumerate}
  \item If $\Span S$ exists, then it is unique.
  \item $\Span S$ exists.
  \item If $S = \{v_1,\dots ,v_n\}$, for some vectors $v_1,\dots,v_n \in \F$, then
        $$\Span S = \{c_1v_1 + c_2v_2 + \dots + c_nv_n \colon c_1,\dots, c_n \in \F\}$$
  \item We like to think that $\Span S$ is the set of all linear combinations of $S$. Therefore we conclude that every vector of the form $c_1v_1 + \dots c_nv_n$, $c_1,\dots, c_n \in \F$ is in $\Span S$.
\end{enumerate}
\end{theorem}
The course notes mention that if $S$ is already a subspace of $V$, then $\Span S = S$.


\subsection{Basis of a Vector Space}
\begin{definition}{Basis of a Vector Space}{}{label}
  Let $V$ be a vector space. A subset $B$ of $V$ is called a \emph{basis} for $V$ if:
  \begin{enumerate}
    \item $B$ is linearly independent.
    \item $\Span B = V$.
  \end{enumerate}
\end{definition}
\textbf{Remark: }Let $V = M_{m\times n}(\F)$, we say that $S = \{E_{ij} \colon 1 \leq n \leq m, 1 \leq j \leq n\}$ is a basis for $V$, called the \emph{standard basis of} $M_{m\times n}(\F)$.

\begin{theorem}{}{label}
  
\end{theorem}


\end{document}






































