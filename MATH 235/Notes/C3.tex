\documentclass[12pt]{article}
\usepackage[utf8]{inputenc} % Para caracteres en español
\usepackage{amsmath,amsthm,amsfonts,amssymb,amscd}
\usepackage{multirow,booktabs}
\usepackage[table]{xcolor}
\usepackage{fullpage}
\usepackage{tikz}
\usepackage{physics}

%%
\usepackage{pgfplots}
\usepgfplotslibrary{polar}
\usepgflibrary{shapes.geometric}
\usetikzlibrary{calc}


\pgfplotsset{my style/.append style={axis x line=middle, axis y line=
           middle, xlabel={$x$}, ylabel={$y$}, axis equal }}
\pgfplotsset{my style-2/.append style={axis x line=middle, axis y line=
           middle, xlabel={$x$}, ylabel={$y$}}}


\usetikzlibrary{arrows, circuits.ee.IEC, positioning}
\usepackage[american voltages, american currents,siunitx]{circuitikz}

\usepackage{lastpage}
\usepackage{enumitem}
\usepackage{fancyhdr}
\usepackage{mathrsfs}
\usepackage{wrapfig}
\usepackage{setspace}
\usepackage{calc}
\usepackage{multicol}
\usepackage{cancel}
\usepackage[retainorgcmds]{IEEEtrantools}
\usepackage[margin=3cm]{geometry}
\usepackage{amsmath}
\newlength{\tabcont}
\setlength{\parindent}{0.0in}
\setlength{\parskip}{0.05in}
\usepackage{empheq}
\usepackage{framed}
\usepackage[most]{tcolorbox}
\usepackage{xcolor}
\colorlet{shadecolor}{orange!15}
\parindent 0in
\parskip 12pt
\geometry{margin=1in, headsep=0.25in}
\theoremstyle{definition}
\newtheorem{defn}{Definition}
\newtheorem{reg}{Rule}
\newtheorem{exer}{Exercise}
\newtheorem{note}{Note}
\usepackage[english]{babel}
\usepackage[protrusion=true,expansion=true]{microtype}  
\usepackage{amsmath,amsfonts,amsthm,amssymb}
\usepackage{graphicx}
\usepackage{tcolorbox}
\usepackage{amssymb}
\tcbuselibrary{theorems}
\newtcbtheorem
  []% init options
  {definition}% name
  {Definition}% title
  {%
    colback=blue!1,
    colframe=blue!75!black,
    fonttitle=\bfseries,
  }% options
  {def}% prefix

\newtcbtheorem
  []% init options
  {lemma}% name
  {Lemma}% title
  {%
    colback=yellow!5,
    colframe=orange!110,
    fonttitle=\bfseries,
  }% options
  {lem}% prefix

\newtcbtheorem
  []% init options
  {theorem}% name
  {Theorem}% title
  {%
    colback=yellow!5,
    colframe=orange!110,
    fonttitle=\bfseries,
  }% options
  {thm}% prefix

\newtcbtheorem
  []% init options
  {solution}% name
  {Solution}% title
  {%
    colback=magenta!5,
    colframe=magenta!150,
    fonttitle=\bfseries,
  }% options
  {not}% prefix

\newtcbtheorem
  []% init options
  {example}% name
  {Example}% title
  {%
    colback=red!5,
    colframe=red!100,
    fonttitle=\bfseries,
  }% options
  {exm}% prefix


\newtcbtheorem
  []% init options
  {algorithm}% name
  {Algorithm} %title
  {%
    colback=blue!5,
    colframe=blue!100,
    fonttitle=\bfseries,
  }% options
  {alg}% prefix

\newtcbtheorem
  []% init options
  {properties}% name
  {Properties}% title
  {%
    colback=green!5,
    colframe=green!35!black,
    fonttitle=\bfseries,
  }% options
  {pro}% prefix



% Sets
\newcommand{\N}{\ensuremath{\mathbb{N}}}
\newcommand{\Z}{\ensuremath{\mathbb{Z}}}
\newcommand{\Q}{\ensuremath{\mathbb{Q}}}
\newcommand{\R}{\ensuremath{\mathbb{R}}}
\newcommand{\C}{\ensuremath{\mathbb{C}}}
\newcommand{\F}{\ensuremath{\mathbb{F}}}
\newcommand{\Mnn}{\ensuremath{M_{n \times n}}}
\newcommand{\sym}{\mathbin{\triangle}}
\newcommand{\stcomp}[1]{{#1}^\complement}

% Operators
\newcommand{\Rarr}{\Rightarrow}
\newcommand{\Larr}{\Leftarrow}
\newcommand{\Harr}{\Leftrightarrow}
\newcommand{\harr}{\leftrightarrow}
\newcommand{\dyx}{\dv{y}{x}}

% Linear Algebra
\newcommand{\vui}{\mathbf{\hat{\textnormal{\bfseries\i}}}}
\newcommand{\vuj}{\mathbf{\hat{\textnormal{\bfseries\j}}}}
\newcommand{\xto}{\xrightarrow} % \xto{R_1 \harr R_2}
\newcommand{\cbm}[2]{\ensuremath{{}_{#1}[I]{}_{#2}}} % change of basis matrix
\DeclareMathOperator{\Proj}{Proj}
\DeclareMathOperator{\Perp}{Perp}
\DeclareMathOperator{\Span}{Span}
\DeclareMathOperator{\Col}{Col}
\DeclareMathOperator{\adj}{adj}

\renewcommand\qedsymbol{$\blacksquare$}

\newcommand{\HRule}[1]{\rule{\linewidth}{#1}}   % Horizontal rule

\makeatletter             % Title
\def\printtitle{%           
    {\centering \@title\par}}
\makeatother                  

\makeatletter             % Author
\def\printauthor{%          
    {\centering \large \@author}}       
\makeatother          
\makeatother  
\title{ \normalsize \textsc{}   % Subtitle
      \\[2.0cm]               % 2cm spacing
      \HRule{0.5pt} \\            % Upper rule
      \LARGE \textbf{\uppercase{Math 116 Handbook}} % Title
      \HRule{2pt} \\ [0.5cm]    % Lower rule + 0.5cm spacing
      \normalsize     % Todays date
    }

\author{
    Abdullah Zubair\\
        \texttt{a6zubair@uwaterloo.ca} \\
}



\begin{document}
\setcounter{section}{2}
\section{Matrix Multiplication, Systems of Linear Equations}
\subsection{Matrix Multiplication}

\begin{definition}{Matrix Multiplication}{label}
   Let $A \in M_{m \times p}(\F)$, $B \in M_{p \times n}$. We define the \emph{matrix product} to be $C = AB \in M_{m\times n}$. The entries $c_{ij}$ of the matrix $C$ are determined by,
   $$c_{ij} = \sum_{k = 1}^n a_{ik}b_{kj} = a_{i1}b_{1j} + a_{i2}b_{2j} + \dots + a_{in}b_{nj}$$
   Where $1 \leq i \leq m$, $1 \leq j \leq n$. 
\end{definition}
\begin{definition}{Indentity Matrix}{label}
    We define $I_n \in M_{n \times n}$ to be,
    $$ I_n = \begin{pmatrix}
       1 & 0 & \dots & 0 \\ 
       0 & 1 & \dots & 0 \\ 
       \vdots & \vdots & \ddots & \vdots \\ 
       0 & 0 & \dots & 1 \\ 
    \end{pmatrix}$$
\end{definition}

\begin{properties}{Matrix Multiplication}{label}
    \begin{enumerate}
        \item 
        For $A \in M_{m \times n}(\F)$, $B \in M_{n\times p}(\F)$, $C \in M_{p \times r}(\F)$,
        \begin{align*}
            (AB)C = A(BC) \tag{Associative}
        \end{align*}
        \item 
        For $A \in M_{m \times n}(\F)$, $B,C\in M_{n\times p}(\F)$,
        \begin{align*}
            A(B + C) = AB + AC \tag{Left distributivity}
        \end{align*}
        \item For $A,B \in M_{m\times n}(\F)$, $C\in M_{n\times p}$
        \begin{align*}
            (A + B)C = AC + BC \tag{Right distributivity}
        \end{align*}
        \item For $A\in M_{m\times n}(\F)$, 
        \begin{align*}
            I_mA = A\\
            AI_n = A
        \end{align*}
        For $Z_m = 0 \in M_{m\times m}, Z_n = 0\in M_{n\times n}$,
        \begin{align*}
            Z_mA = Z_m\\
            AZ_n = Z_n
        \end{align*}
        \item For $A\in M_{m\times n}(\F), B \in M_{n\times p})(\F)$, and $c\in \F$,
        $$c(AB) = (cA)B = A(cB)$$

    \end{enumerate}
\end{properties}

\subsection{Solving Systems of Linear Equations}
\begin{definition}{Systems of Linear Equations, Homogenous and Inhomogenousand}{label}
  A \emph{system of linear equations} is a simultaneous set of equations of the form
  $$\begin{cases}
    a_{11}x_1 + a_{12}x_2 + \dots + x_{1n}x_n = b_1\\
    a_{21}x_1 + a_{22}x_2 + \dots + x_{2n}x_n = b_2\\
    \vdots\\
    a_{m1}x_1 + a_{m2}x_2 + \dots + x_{mn}x_n = b_m\\
  \end{cases}$$
  Where the quantities $x_1,\dots,x_n$ are the \emph{unknowns} and the numbers $a_{ij}$ and $b_i \in \mathbb F$. We would like to determine which values $x_1,\dots,x_n \in  \mathbb F$ satisfy all $m$ equations. Such an assignment is known as a \emph{solution} to the system.\\

  If $b_1 = \dots = b_n = 0$ then the system is refereed to as a \emph{homogenous}. Otherwise, the system is called \emph{inhomogeneous}
  
\end{definition}
\begin{definition}{Elementary Row Operations}{label}
  Suppose $A\in M_{m\times n}(\F)$. An \emph{elementary row operation} is an operation preformed on the entries of $A$. The three types of operations are;
  \begin{enumerate}
    \item Swap two distinct rows of $A$, $\text{Row}_i \leftrightarrow \text{Row}_j$
    \item Multiply all entires of a row $\text{Row}_i$ by a \emph{non-zero} scalar $c\in \mathbb F$,  $\text{Row}_i \rightarrow c\text{Row}_i$
    \item Add a multiple of some row $\text{Row}_j$ to row $\text{Row}_i$ $(i \neq j)$, $\text{Row}_i \rightarrow \text{Row}_i + c\text{Row}_j$
  \end{enumerate}
Note that all of the operations are \emph{invertible}. For example, if $A_1$ was obtained from $A$ by preforming one of the three operations, then there exists an operation of the same type to reobtain $A$. 
\end{definition}

\begin{definition}{Row equiavalence}{label}
  If $B$ is a matrix obtained from $A$ by preforming some finite number of operations, then we say that $A$ and $B$ are \emph{row equivalent}. 
\end{definition}
The importance of row equivalency is the fact that the solutions to $A\vb x = \vb 0$ is preserved in $B\vb x = \vb 0$, this is primarily due to the fact that row operations are reversible and that $A$ can be obtained from $B$ by preforming some finite number of row operations.
\begin{definition}{REF, RREF}{label}
  Suppose $R\ in M_{m\times n}(\F)$. We say that $R$ is in \emph{row echelon form}(REF) if:
  \begin{enumerate}
    \item All zero rows are below all non-zero rows
    \item In any non-zero row, the \emph{pivot} (first non-zero entry in the row) has only zero entires below in the same column.
  \end{enumerate}
  We say that $R$ is in \emph{reduced row echelon form}(RREF) if:
  \begin{enumerate}
    \item All zero rows are below all non-zero rows
    \item In any non-zero row, the \emph{pivot}(first non-zero entry in the row) is $1$ and it is the only non-zero entry in the corresponding column. 
  \end{enumerate}
  
\end{definition}


\begin{theorem}{}{label}
  Suppose $A \in M_{m\times n}(\F)$, and let $\vb b$ be a non-zero vector in $\F^m$.
  \begin{enumerate}
    \item The set of solutions $\vb x\in F^n$ to the homogenous system $A\vb x = \vb 0$ forms a subspace of $\F^n$.
    \item If $\vb x_p$ is some particular solution to the inhomogeneous system $A\vb x = \vb b$, then every solution to the inhomogeneous system has the form $\vb y + \vb x_p$, where $A\vb y = \vb 0$. Conversely, every vector of the form $\vb y + \vb x_p$, where $A\vb y = \vb 0$, a solution to $A\vb x = \vb b$ 
  \end{enumerate}
\end{theorem}

\end{document}


















