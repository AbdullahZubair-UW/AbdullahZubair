\documentclass[12pt]{article} 

\usepackage{fullpage}
\usepackage{bookmark}
\usepackage{amsmath}
\usepackage{amssymb}
\usepackage[dvipsnames]{xcolor}
\usepackage{hyperref} % for the URL
\usepackage[shortlabels]{enumitem}
\usepackage{mathtools}
\usepackage[most]{tcolorbox}
\usepackage[amsmath,standard,thmmarks]{ntheorem} 
\usepackage{physics}
\usepackage{pst-tree} % for the trees
\usepackage{verbatim} % for comments, for version control
\usepackage{tabu}
\usepackage{tikz}
\usepackage{float}
\usepackage{siunitx}
\usepackage{physunits}
\usepackage{pgfplots}

% From the Plot video

\usepackage[LGR,T1]{fontenc}
\usepackage[utf8]{inputenc}
\usepackage{lmodern}
\usepackage{microtype}
\usepackage{upgreek}
\usepackage[misc]{ifsym}

\usepackage{pgfplots}
	\usetikzlibrary{
		calc,
		patterns,
		positioning
	}
	\pgfplotsset{
		compat=1.16,
		samples=200,
		clip=false,
		my axis style/.style={
			axis x line=middle,
			axis y line=middle,
			legend pos=outer north east,
			axis line style={
				->,
			},
			legend style={
				font=\footnotesize
			},
			label style={
				font=\footnotesize
			},
			tick label style={
				font=\footnotesize
			},
			xlabel style={
				at={
					(ticklabel* cs:1)
				},
				anchor=west,
				font=\footnotesize,
			},
			ylabel style={
				at={
					(ticklabel* cs:1)
				},
				anchor=west,
				font=\footnotesize,
			},
			xlabel=$t$,
			ylabel=$\vec v (\m \tx{[East]})$
		},
	}
	\tikzset{
		>=stealth
	}


    \pgfplotsset{my style/.append style={axis x line=middle, axis y line=
           middle, xlabel={$t$}, axis equal }}

%%% Tables and figures packages

\usepackage{float}
\usepackage{caption}
	\captionsetup{
		format=plain,
		labelfont=bf,
		font=small,
		justification=centering
	}
	
%%% Numbers and sets

\newcommand{\E}{\mathrm{e}}
\newcommand{\tx}[1]{\text{#1}}


% floor, ceiling, set
\DeclarePairedDelimiter{\ceil}{\lceil}{\rceil}
\DeclarePairedDelimiter{\floor}{\lfloor}{\rfloor}
\DeclarePairedDelimiter{\set}{\lbrace}{\rbrace}
\DeclarePairedDelimiter{\iprod}{\langle}{\rangle}

\DeclareMathOperator{\Int}{int}
\DeclareMathOperator{\mean}{mean}

% commonly used sets
\newcommand{\R}{\mathbb{R}}
\newcommand{\Nat}{\mathbb{N}}
\newcommand{\Q}{\mathbb{Q}}
\renewcommand{\P}{\mathbb{P}}

\newcommand{\sset}{\subseteq}


\theoremstyle{break}
\theorembodyfont{\upshape}

\newtheorem{thm}{Theorem}[subsection]
\tcolorboxenvironment{thm}{
enhanced jigsaw,
colframe=Dandelion,
colback=White!90!Dandelion,
drop fuzzy shadow east,
rightrule=2mm,
sharp corners,
before skip=10pt,after skip=10pt
}

\newtheorem{cor}{Corollary}[thm]
\tcolorboxenvironment{cor}{
boxrule=0pt,
boxsep=0pt,
colback={White!90!RoyalPurple},
enhanced jigsaw,
borderline west={2pt}{0pt}{RoyalPurple},
sharp corners,
before skip=10pt,
after skip=10pt,
breakable
}

\newtheorem{algo}[thm]{Algorithm}
\tcolorboxenvironment{algo}{
enhanced jigsaw,
colframe=Red,
colback={White!95!Red},
rightrule=2mm,
sharp corners,
before skip=10pt,after skip=10pt
}

\newtheorem{ex}[thm]{Example}
\tcolorboxenvironment{ex}{% from ntheorem
blanker,left=5mm,
sharp corners,
before skip=10pt,after skip=10pt,
borderline west={2pt}{0pt}{Green}
}

\newtheorem*{pf}{Proof}
\tcolorboxenvironment{pf}{% from ntheorem
breakable,blanker,left=5mm,
sharp corners,
before skip=10pt,after skip=10pt,
borderline west={2pt}{0pt}{NavyBlue!80!white}
}


\newtheorem*{soln}{Solution}
\tcolorboxenvironment{soln}{% from ntheorem
breakable,blanker,left=5mm,
sharp corners,
before skip=10pt,after skip=10pt,
borderline west={2pt}{0pt}{NavyBlue!80!white}
}

\newtheorem{defn}{Definition}[subsection]
\tcolorboxenvironment{defn}{
enhanced jigsaw,
colframe=Cerulean,
colback=White!90!Cerulean,
drop fuzzy shadow east,
rightrule=2mm,
sharp corners,
before skip=10pt,after skip=10pt
}

\newtheorem{prop}[thm]{Proposition}
\tcolorboxenvironment{prop}{
boxrule=0pt,
boxsep=0pt,
colback={White!90!Green},
enhanced jigsaw,
borderline west={2pt}{0pt}{Green},
sharp corners,
before skip=10pt,
after skip=10pt,
breakable
}

\setlength\parindent{0pt}
\setlength{\parskip}{2pt}


\begin{document}
\let\ref\Cref
\section{Sets}
\subsection{Introduction}
\begin{defn}
\textbf{Sets} are defined to be a collection of \emph{objects} composed inside a pair of braces {}.
\end{defn}
To define what we mean by an $\emph{object}$ can be complicated, and hence I will refer to objects as anything that has been previously defined or "tangible" (although even this can get a little philosophical). For example if the object is an integer, then we can build a set with some integers, take $\{2,3,44,5\}$ as an example or take the following tangible objects $\clubsuit, \heartsuit, \triangle$ and build a set with them $\{\clubsuit, \heartsuit, \triangle \}$. There are some key properties of sets to note. $\textbf{Order does not matter}$, meaning any rearrangement of the objects in a set yields the same set, for example we say that $\{1,2,3,4,5\} = \{2,3,4,5,1\} = \{1,2,3,5,4\}$, etc. Also, $\textbf{duplicates are not allowed}$ so whenever we observe a duplicate object, we immediately remove it and yield an equivalent set, so $\{1,2,2,3,4\} = \{1,2,3,4\}$. We say that the $\textbf{cardinality}$ of a set $\mathcal{S}$ is the number of elements (or objects) in the set, and denote the quantity as $|\mathcal{S}|$, for example if $\mathcal{S} = \{1,2,3,4,5\}$ then $|\mathcal{S}| = 5$.

\begin{defn}
    We denote $\emptyset$ as the set with no elements, and call it the $\textbf{empty set}$. This implies $|\emptyset| = 0$.
  \end{defn}
  
  
\textbf{Notation:} If some element $x$ is contained within a set $\mathcal{S}$, then we say that $x$ is an element of $\mathcal{S}$ and write $x\in S$. Consequently, if some element $y$ is \textbf{not} an element of $\mathcal{S}$, then we say that $y$ is not an element of $\mathcal{S}$ and write $y\not\in \mathcal{S}$.


%%%%%%%%%%%%%%%%%%%%%%%%%%%%%%%%%%%%%%%%%%%%%%%%%%%%%%%%%%%%%%%%%%%%%%%%%%%%%%%%%%%%%%%%%%%%%%%%%%%%%%%%%%%%%%%%%%%%%%%%%%%%%%%%%%%%%%%%%%%%%%%%%%%%%%%%%%%%%%%%%%%%%%%%%%%%%%%%%%%%%%%%%%%%%%%%%%%%%%%%%%%%%%%%%%%%%%%%%%%%%%%%%%%%%%%%%%%%%%%%%%%%%%%%%%%%%%%%%%%%%%%%%%%%%%%%%%%%%%%%%%%%%%%%%%%%%%%%%%%%%%%%%%

\newpage


\subsection{Common Sets}
There are a few common recurring sets that are the building blocks for the objects we will manipulate throughout this book. We list them here, (note that the $\dots$ notation indicates a continuation following the logical pattern)




\begin{enumerate}
\item $\mathbb Z$ denotes the set of all integers $\mathbb Z = \{\dots,-2,-1,0,1,2,\dots \}$.
\item $\mathbb N$ denotes the set of all \emph{positive} integers $\mathbb N = \{1,2,3,\dots \}$.
\item $\mathbb R$ denotes the set of all real numbers (rational or irrational). 
\item $\mathbb Q$ denotes the set of all rational numbers.
\item $\mathbb Z^+$ denotes the set of all non-negative integers $\mathbb Z^+ = \{0,1,2,3,\dots\}$\\
\textbf{Remark:} Some texts will not allow $0$ to be apart of $\mathbb Z^+$.
\item $\mathbb R^+$ denotes the set of all non-negative positive real numbers.
\item $\mathbb Z^-$ denotes the set of all negative integers $\mathbb Z^+ = \{-1,-2,-3,\dots\}$
\item $\mathbb R^-$ denotes the set of all negative real numbers.
\end{enumerate}



\textbf{Remark: }In some very specific math subjects we like to say that $\N$ includes $0$ as well, this can be particularly useful whenever there is some sort of correspondence to Computer Science. 

The \textbf{universe of discourse} denoted $\mathcal{U}$, is the set of all objects we may be interested in a given scenario. In this book, we are mostly always working with the set $\R$, and hence the universe of discourse will almost always be $\mathcal{U} = \R$. (There may be a few special cases were we explicitly differentiate). 





\begin{defn}
We say that the set $\{x\in \mathcal{U} \colon \textbf{statement} \}$ is the set of all elements $x$ in $\mathcal{U}$ such that the \textbf{statement} is true for $x$. (The semicolon means "such that", some texts will use a $|$ instead)
\end{defn}





\end{document}
