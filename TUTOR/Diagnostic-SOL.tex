% Prepared by Calvin Kent
%
% Assignment Template v19.02
%
%%% 20xx0x/MATHxxx/Crowdmark/Ax
%
\documentclass[11pt]{article} %
\usepackage{CKpreamble}
\usepackage{CKassignment}
\usepackage{tkz-euclide}


\begin{document}
    \pagenumbering{arabic}
    % Start of class settings ...
    \renewcommand*{\coursecode}{Physics Diagnostic} % renew course code
    \renewcommand*{\assgnnumber}{1} % renew assignment number
    \renewcommand*{\submdate}{August 13, 2021} % renew the date
    \renewcommand*{\studentfname}{Abdullah} % Student first name
    \renewcommand*{\studentlname}{Zubair} % Student last name
    %\renewcommand*{\studentnum}{SNumber} % Student number

    \renewcommand\qedsymbol{$\blacksquare$}
    \setfigpath
    % End of class settings 
    \pagestyle{crowdmark}
    \newgeometry{left=18mm, right=18mm, top=22mm, bottom=22mm} % page is set to default values
    \fancyhfoffset[L,O]{0pt} % header orientation fixed
    % End of class settings
    %%% Note to user:
    % CTRL + F <CHANGE ME:> (without the angular brackets) in CKpreamble to specify graphics paths accordingly.
    % The command \circled[]{} accepts one optional and one mandatory argument.
    % Optional argument is for the size of the circle and mandatory argument is for its contents.
    % \circled{A} produces circled A, with size drawn for letter A. \circled[TT]{A} produces circled A with size drawn for TT.
    % https://github.com/CalvinKent/My-LaTeX
    %%%
    % Crowdmark assignment start
\begin{qstn}[1] % qnumber, qname, qpoints
Solve the following equations
\begin{enumerate}
\item $2x + 4 = 12$

\begin{soln}
\begin{align*}
2x + 4 &= 12\\
2x &= 12 - 4\\
2x &= 8\\
x &= 4
\end{align*}
\end{soln}

\item $2x^2 + 3x + 6 = 12$
\begin{soln}
\begin{align*}
2x^2 + 3x + 6 &= 12\\
2x^2 + 3x + 6 - 12 &= 0\\
2x^2 + 3x - 6 &= 0\\
x &= \frac{-3 \pm \sqrt{(3)^2 - 4(2)(-6)}}{2(2)} \tag{\text{Quadratic Formula}}\\
x &=  \frac{-3 \pm \sqrt{57}}{4}\\
\implies x &= \frac{-3 + \sqrt{57}}{4}, \frac{-3 - \sqrt{57}}{4}\\
\text{(OR in decimal)} x &= -2.6375, 1.137
\end{align*}
\end{soln}

\item $\sin(30)\cdot \cos(60) = \frac{3}{x}$

\begin{soln}
\begin{align*}
\sin(30)\cdot \cos(60) &= \frac{3}{x}\\
(\frac{1}{2})\cdot(\frac{1}{2}) &= \frac{3}{x}\\
\frac{1}{4} &= \frac{3}{x}\\
x &= 12
\end{align*}
\end{soln}

\item $10x^2 - 20x + 12 = 8x^2 + 16x + 4$

\begin{soln}
\begin{align*}
10x^2 - 20x - 12 &= 8x^2 + 16x + 4 \\
10x^2 - 8x^2 - 20x - 16x + 12 - 4 &= 0\\
2x^2 -36x + 8 &= 0 \\
2(x^2 - 18x + 4) &= 0\\
x^2 - 18x + 4 &= 0\\
x &= \frac{-(-18) \pm \sqrt{(-18)^2 - 4(1)(4)}}{2(1)}\\
x &= \frac{18 \pm \sqrt{240}}{2}\\
\implies x &= \frac{18 + \sqrt{308}}{2}, \frac{18 - \sqrt{308}}{2}\\
\text{(OR in decimal)} x &= 0.2250, 17.775
\end{align*}

\end{soln}



\end{enumerate}
  \end{qstn}

\begin{qstn}[2]
Each question below will ask you to solve for a given variable.

\begin{enumerate}
\item Solve for $a$ $$F = ma$$

\begin{soln}

$$a = \frac{F}{m}$$

\end{soln}


\item Solve for $T$ $$PV = NRT$$
\begin{soln}
\begin{align*}
PV &= NRT\\
T &= \frac{PV}{NR}
\end{align*}
\end{soln}



\item Solve for $t$ $$ d = \left(\frac{v_f + v_i}{2}\right)t$$

\begin{soln}
\begin{align*}
d &= \left(\frac{v_f + v_i}{2}\right)t \\
d &= \frac{\left(v_f + v_i \right)t}{2} \\
2d &= (v_f + v_i)t\\
t& = \frac{2d}{v_f + v_i}
\end{align*}

\end{soln}

\newpage
\item Solve for $a$ $$\Delta d = v_it + \frac{1}{2}at^2$$

\begin{soln}
\begin{align*}
\Delta d &= v_it + \frac{1}{2}at^2\\
\Delta d - v_it &= \frac{1}{2}at^2 \\
2(\Delta d - v_it) &= at^2 \\
a &= \frac{2(\Delta d - v_it)}{t^2}
\end{align*}

\end{soln}


\item Solve for $v_f$ $$v_f^2 = v_i^2 + 2a\Delta d$$

\begin{soln}
\begin{align*}
v_f &= \pm \sqrt{v_i^2 + 2a\Delta d} \tag{\text{$\pm$ IMPORTANT}}
\end{align*}

\end{soln}



\end{enumerate}
\end{qstn}


\begin{qstn}[2]
Determine the length of the unkown side $X$ for the triangle below

\begin{center}
\begin{tikzpicture}[thick]
\coordinate (O) at (0,0);
\coordinate (A) at (14,0);
\coordinate (B) at (14,5);
\draw (O)--(A)--(B)--cycle;

\tkzLabelSegment[below=2pt](O,A){$16$}
\tkzLabelSegment[left=12pt](O,B){\textit{$32$}}
\tkzLabelSegment[right=2pt](A,B){\textit{$X$}}

\tkzLabelAngle[pos = 1.5](A,O,B){$\gamma$}


\end{tikzpicture}
\end{center}


\begin{soln}
We proceed with the Pythagorean Theorem,
\begin{align*}
X^2 + 16^2 &= 32^2\\
X^2 &= 32^2 - 16^2\\
X^2 &= 768 \\ 
X &= +\sqrt{768} \tag{\text{Negative side is illogical}}
\end{align*}



\end{soln}



\end{qstn}

\begin{qstn}[3] Determine the angle $\gamma$

\begin{center}
\begin{tikzpicture}[thick]
\coordinate (O) at (0,0);
\coordinate (A) at (14,0);
\coordinate (B) at (14,5);
\draw (O)--(A)--(B)--cycle;

\tkzLabelSegment[below=2pt](O,A){$16$}
\tkzLabelSegment[left=12pt](O,B){\textit{$32$}}

\tkzLabelAngle[pos = 1.5](A,O,B){$\gamma$}


\end{tikzpicture}
\end{center}



\begin{soln}

\begin{align*}
\cos (\gamma) &= \frac{16}{32}\\
\cos (\gamma) &= \frac{1}{2}\\
\implies \gamma &= 60
\end{align*}

\end{soln}




\end{qstn}


\end{document}
%   \begin{figure}[H]
%   \centering
%   \includegraphics[width=0.75\linewidth]{p}
%   \caption{caption.\label{fig:}}
%   \end{figure}
