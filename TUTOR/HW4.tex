% Prepared by Calvin Kent
%
% Assignment Template v19.02
%
%%% 20xx0x/MATHxxx/Crowdmark/Ax
%
\documentclass[12pt]{article} %
\usepackage{CKpreamble}
\usepackage{CKassignment}
\usepackage{tkz-euclide}
\usepackage{physunits}
\usepackage{physics}
\usepackage{lmodern}
\usepackage{microtype}
\usepackage{upgreek}
\usepackage[misc]{ifsym}

%%% Maths and science packages

\usepackage{amsmath,amsthm,amssymb}
\usepackage{pgfplots}
	\usetikzlibrary{
		calc,
		patterns,
		positioning
	}
	\pgfplotsset{
		compat=1.16,
		samples=200,
		clip=false,
		my axis style/.style={
			axis x line=middle,
			axis y line=middle,
			legend pos=outer north east,
			axis line style={
				->,
			},
			legend style={
				font=\footnotesize
			},
			label style={
				font=\footnotesize
			},
			tick label style={
				font=\footnotesize
			},
			xlabel style={
				at={
					(ticklabel* cs:1)
				},
				anchor=west,
				font=\footnotesize,
			},
			ylabel style={
				at={
					(ticklabel* cs:1)
				},
				anchor=west,
				font=\footnotesize,
			},
			xlabel= $t$,
			ylabel=$\vec d (\m \tx{[East]})$
		},
	}
	\tikzset{
		>=stealth
	}

%%% Tables and figures packages

\usepackage{float}
\usepackage{caption}
	\captionsetup{
		format=plain,
		labelfont=bf,
		font=small,
		justification=centering
	}
	
%%% Numbers and sets

\newcommand{\E}{\mathrm{e}}

\newcommand{\tx}[1]{\text{#1}}

\begin{document}
    \pagenumbering{arabic}
    % Start of class settings ...
    \renewcommand*{\coursecode}{Physics Homework} % renew course code
    \renewcommand*{\assgnnumber}{4} % renew assignment number
    \renewcommand*{\submdate}{August 26, 2021} % renew the date
    \renewcommand*{\studentfname}{Abdullah} % Student first name
    \renewcommand*{\studentlname}{Zubair} % Student last name
    %\renewcommand*{\studentnum}{SNumber} % Student number

    \renewcommand\qedsymbol{$\blacksquare$}
    \setfigpath
    % End of class settings 
    \pagestyle{crowdmark}
    \newgeometry{left=18mm, right=18mm, top=22mm, bottom=22mm} % page is set to default values
    \fancyhfoffset[L,O]{0pt} % header orientation fixed
    % End of class settings
    %%% Note to user:
    % CTRL + F <CHANGE ME:> (without the angular brackets) in CKpreamble to specify graphics paths accordingly.
    % The command \circled[]{} accepts one optional and one mandatory argument.
    % Optional argument is for the size of the circle and mandatory argument is for its contents.
    % \circled{A} produces circled A, with size drawn for letter A. \circled[TT]{A} produces circled A with size drawn for TT.
    % https://github.com/CalvinKent/My-LaTeX
    %%%
    % Crowdmark assignment start
\begin{qstn}[1] % qnumber, qname, qpoints
    Answer the following True/False questions (\textbf{Assume [East] is positive})
    \begin{enumerate}
        \item An object under uniform motion has a
            \begin{enumerate}[label = (\alph*)]
                \item Non-zero average acceleration in the positive direction (T / F) 
                \item Zero average acceleration (T / F)
            \end{enumerate}
        \item De-acceleration is just acceleration in the same direction of motion (T / F)
        \item Suppose a Velocity V. Time plot is represented by $y = 2x + 4$,
            \begin{enumerate}[label = (\alph*)]
                \item The average acceleration is uniform (T / F)
                \item The initial velocity of the body at $t = 0$ was $\vec v_i = +4 \m / \s$ (T / F)
                \item The displacement over the time interval $[0,2]$ was $\Delta \vec d = +12$m (T / F)
                \item The average acceleration is $\vec a_{av} = +2\m / \s^2$ (T / F)
            \end{enumerate}
        \item A secant line on a Velocity V. Time graph over the interval $[t_1,t_2]$ gives me the instantaneous acceleration over the time interval $[t_1,t_2]$. (T / F)
        \item Suppose a Position V. Time plot is represented by $y = x^2 + 4$. Then,
            \begin{enumerate}[label = (\alph*)]
                \item Slowing down in the positive direction. (T / F)
                \item The object is experiencing uniform motion. (T / F)
                \item The object \emph{may} be experiencing uniform acceleration (T / F).
                \item The initial position vector of the object at $t = 0$ is $\vec d_i = 2 \m$
            \end{enumerate}
        

        
        \item Suppose a Velocity V. Time plot is represented by $y = -x + 3$, then the displacement over the time interval $[0,8]$ is $\Delta \vec d = +0 \m$. (T / F)

    \end{enumerate}

 \end{qstn}


\end{document}
