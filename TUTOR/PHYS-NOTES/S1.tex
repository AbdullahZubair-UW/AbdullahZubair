\documentclass[12pt]{article} 

\usepackage{fullpage}
\usepackage{bookmark}
\usepackage{amsmath}
\usepackage{amssymb}
\usepackage[dvipsnames]{xcolor}
\usepackage{hyperref} % for the URL
\usepackage[shortlabels]{enumitem}
\usepackage{mathtools}
\usepackage[most]{tcolorbox}
\usepackage[amsmath,standard,thmmarks]{ntheorem} 
\usepackage{physics}
\usepackage{pst-tree} % for the trees
\usepackage{verbatim} % for comments, for version control
\usepackage{tabu}
\usepackage{tikz}
\usepackage{float}
\usepackage{siunitx}
\usepackage{physunits}



% floor, ceiling, set
\DeclarePairedDelimiter{\ceil}{\lceil}{\rceil}
\DeclarePairedDelimiter{\floor}{\lfloor}{\rfloor}
\DeclarePairedDelimiter{\set}{\lbrace}{\rbrace}
\DeclarePairedDelimiter{\iprod}{\langle}{\rangle}

\DeclareMathOperator{\Int}{int}
\DeclareMathOperator{\mean}{mean}

% commonly used sets
\newcommand{\R}{\mathbb{R}}
\newcommand{\Nat}{\mathbb{N}}
\newcommand{\Q}{\mathbb{Q}}
\renewcommand{\P}{\mathbb{P}}

\newcommand{\sset}{\subseteq}


\theoremstyle{break}
\theorembodyfont{\upshape}

\newtheorem{thm}{Theorem}[subsection]
\tcolorboxenvironment{thm}{
enhanced jigsaw,
colframe=Dandelion,
colback=White!90!Dandelion,
drop fuzzy shadow east,
rightrule=2mm,
sharp corners,
before skip=10pt,after skip=10pt
}

\newtheorem{cor}{Corollary}[thm]
\tcolorboxenvironment{cor}{
boxrule=0pt,
boxsep=0pt,
colback={White!90!RoyalPurple},
enhanced jigsaw,
borderline west={2pt}{0pt}{RoyalPurple},
sharp corners,
before skip=10pt,
after skip=10pt,
breakable
}

\newtheorem{lem}[thm]{Lemma}
\tcolorboxenvironment{lem}{
enhanced jigsaw,
colframe=Red,
colback={White!95!Red},
rightrule=2mm,
sharp corners,
before skip=10pt,after skip=10pt
}

\newtheorem{ex}[thm]{Example}
\tcolorboxenvironment{ex}{% from ntheorem
blanker,left=5mm,
sharp corners,
before skip=10pt,after skip=10pt,
borderline west={2pt}{0pt}{Green}
}

\newtheorem*{pf}{Proof}
\tcolorboxenvironment{pf}{% from ntheorem
breakable,blanker,left=5mm,
sharp corners,
before skip=10pt,after skip=10pt,
borderline west={2pt}{0pt}{NavyBlue!80!white}
}


\newtheorem*{soln}{Solution}
\tcolorboxenvironment{soln}{% from ntheorem
breakable,blanker,left=5mm,
sharp corners,
before skip=10pt,after skip=10pt,
borderline west={2pt}{0pt}{NavyBlue!80!white}
}

\newtheorem{defn}{Definition}[subsection]
\tcolorboxenvironment{defn}{
enhanced jigsaw,
colframe=Cerulean,
colback=White!90!Cerulean,
drop fuzzy shadow east,
rightrule=2mm,
sharp corners,
before skip=10pt,after skip=10pt
}

\newtheorem{prop}[thm]{Proposition}
\tcolorboxenvironment{prop}{
boxrule=0pt,
boxsep=0pt,
colback={White!90!Green},
enhanced jigsaw,
borderline west={2pt}{0pt}{Green},
sharp corners,
before skip=10pt,
after skip=10pt,
breakable
}

\setlength\parindent{0pt}
\setlength{\parskip}{2pt}


\begin{document}
\let\ref\Cref

\title{\bf{Physics 11 : Course Notes}}
\date{\today}
\author{Abdullah Zubair}

\maketitle
\newpage
\tableofcontents
\listoffigures
\listoftables
\newpage

\section{Introduction}
This course will provide students with a very brief introduction to Newtonian mechanics, a branch of physics
which explains the motion of moving bodies. Throughout this course students will encounter concepts such as
Newtons laws, Energy conservation, Nuclear physics, Sounds and waves as well as Electromagnetic induction.
Students are expected to have a developed background in mathematics as math is the concepts, ideas and relations in physics are described mathematically. Hence, students should be given an
language of physics, all
appropriate math pre-assessment testing basic concepts from algebra and trigonometry.


\newpage

\section{Units}
Unlike mathematics, almost all quantities in physics are expressed in some units in a manner to which which
allows us to make comparisons and understand the magnitude of a given quantity. For example in mathematics if
I tell you that object $A = 50$ and $B = 20$, then you may immediately conclude that $A > B$, however in
physics you may be skeptical about $A > B$ because what if $A = 50 \si{g}$ and $B = 20 \si{kg}$. While solving problems throughout this course, you may encounter situations which require you to convert a quantity in a particular unit to another. For example, we almost always work in $\si{kg}$ units in this course and therefore you may need to preform some sort of conversion operation to obtain a final quantity in $\si{kg}$. We will discuss below the general technique for approaching such problems, remember to always refer to the \textbf{Conversion Table} whenever solving such problems.

\begin{ex}
I have a box weighing at $1250\si{g}$. How much does this box weight in $\si{kg}$?
\end{ex}
\begin{soln}
We proceed by cancelling units,
$$1250\gm\cdot\left(\frac{1\kg}{1000\gm} \right) = \left(\frac{1250}{1000}\right)\kg = 1.250\kg$$
\end{soln}


\begin{ex}
My vehicle is currently driving at a speed of $\text{speed} = 60\frac{\km}{\h}$, what is my speed in $\frac{\m}{\s}$?
\end{ex}
\begin{soln}
We proceed by cancelling units,
	$$ 60\frac{\km}{\h} \cdot \left(\frac{1000\m}{1\km}\right) \cdot \left(\frac{1\h}{3600\s} \right) = 16.667 \frac{\m}{\s}$$
\end{soln}








\end{document}