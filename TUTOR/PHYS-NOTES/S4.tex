\documentclass[12pt]{article} 

\usepackage{fullpage}
\usepackage{bookmark}
\usepackage{amsmath}
\usepackage{amssymb}
\usepackage[dvipsnames]{xcolor}
\usepackage{hyperref} % for the URL
\usepackage[shortlabels]{enumitem}
\usepackage{mathtools}
\usepackage[most]{tcolorbox}
\usepackage[amsmath,standard,thmmarks]{ntheorem} 
\usepackage{physics}
\usepackage{pst-tree} % for the trees
\usepackage{verbatim} % for comments, for version control
\usepackage{tabu}
\usepackage{tikz}
\usepackage{float}
\usepackage{siunitx}
\usepackage{physunits}

% From the Plot video

\usepackage[LGR,T1]{fontenc}
\usepackage[utf8]{inputenc}
\usepackage{lmodern}
\usepackage{microtype}
\usepackage{upgreek}
\usepackage[misc]{ifsym}

\usepackage{pgfplots}
	\usetikzlibrary{
		calc,
		patterns,
		positioning
	}
	\pgfplotsset{
		compat=1.16,
		samples=200,
		clip=false,
		my axis style/.style={
			axis x line=middle,
			axis y line=middle,
			legend pos=outer north east,
			axis line style={
				->,
			},
			legend style={
				font=\footnotesize
			},
			label style={
				font=\footnotesize
			},
			tick label style={
				font=\footnotesize
			},
			xlabel style={
				at={
					(ticklabel* cs:1)
				},
				anchor=west,
				font=\footnotesize,
			},
			ylabel style={
				at={
					(ticklabel* cs:1)
				},
				anchor=west,
				font=\footnotesize,
			},
			xlabel=$t$,
			ylabel=$\vec v (\m \tx{[East]})$
		},
	}
	\tikzset{
		>=stealth
	}


    \pgfplotsset{my style/.append style={axis x line=middle, axis y line=
           middle, xlabel={$t$}, axis equal }}

%%% Tables and figures packages

\usepackage{float}
\usepackage{caption}
	\captionsetup{
		format=plain,
		labelfont=bf,
		font=small,
		justification=centering
	}
	
%%% Numbers and sets

\newcommand{\E}{\mathrm{e}}
\newcommand{\tx}[1]{\text{#1}}


% floor, ceiling, set
\DeclarePairedDelimiter{\ceil}{\lceil}{\rceil}
\DeclarePairedDelimiter{\floor}{\lfloor}{\rfloor}
\DeclarePairedDelimiter{\set}{\lbrace}{\rbrace}
\DeclarePairedDelimiter{\iprod}{\langle}{\rangle}

\DeclareMathOperator{\Int}{int}
\DeclareMathOperator{\mean}{mean}

% commonly used sets
\newcommand{\R}{\mathbb{R}}
\newcommand{\Nat}{\mathbb{N}}
\newcommand{\Q}{\mathbb{Q}}
\renewcommand{\P}{\mathbb{P}}

\newcommand{\sset}{\subseteq}


\theoremstyle{break}
\theorembodyfont{\upshape}

\newtheorem{thm}{Theorem}[subsection]
\tcolorboxenvironment{thm}{
enhanced jigsaw,
colframe=Dandelion,
colback=White!90!Dandelion,
drop fuzzy shadow east,
rightrule=2mm,
sharp corners,
before skip=10pt,after skip=10pt
}

\newtheorem{cor}{Corollary}[thm]
\tcolorboxenvironment{cor}{
boxrule=0pt,
boxsep=0pt,
colback={White!90!RoyalPurple},
enhanced jigsaw,
borderline west={2pt}{0pt}{RoyalPurple},
sharp corners,
before skip=10pt,
after skip=10pt,
breakable
}

\newtheorem{algo}[thm]{Algorithm}
\tcolorboxenvironment{algo}{
enhanced jigsaw,
colframe=Red,
colback={White!95!Red},
rightrule=2mm,
sharp corners,
before skip=10pt,after skip=10pt
}

\newtheorem{ex}[thm]{Example}
\tcolorboxenvironment{ex}{% from ntheorem
blanker,left=5mm,
sharp corners,
before skip=10pt,after skip=10pt,
borderline west={2pt}{0pt}{Green}
}

\newtheorem*{pf}{Proof}
\tcolorboxenvironment{pf}{% from ntheorem
breakable,blanker,left=5mm,
sharp corners,
before skip=10pt,after skip=10pt,
borderline west={2pt}{0pt}{NavyBlue!80!white}
}


\newtheorem*{soln}{Solution}
\tcolorboxenvironment{soln}{% from ntheorem
breakable,blanker,left=5mm,
sharp corners,
before skip=10pt,after skip=10pt,
borderline west={2pt}{0pt}{NavyBlue!80!white}
}

\newtheorem{defn}{Definition}[subsection]
\tcolorboxenvironment{defn}{
enhanced jigsaw,
colframe=Cerulean,
colback=White!90!Cerulean,
drop fuzzy shadow east,
rightrule=2mm,
sharp corners,
before skip=10pt,after skip=10pt
}

\newtheorem{prop}[thm]{Proposition}
\tcolorboxenvironment{prop}{
boxrule=0pt,
boxsep=0pt,
colback={White!90!Green},
enhanced jigsaw,
borderline west={2pt}{0pt}{Green},
sharp corners,
before skip=10pt,
after skip=10pt,
breakable
}

\setlength\parindent{0pt}
\setlength{\parskip}{2pt}


\begin{document}
\let\ref\Cref
\section{Acceleration}
\begin{defn}
\textbf{Acceleration}, $\vec a_{av}$, refers to the rate of change of velocity, or in other words the ratio of the change of velocity to the time elapsed. (\textbf{Units:} $m / \s^2$)
$$\vec a_{av} = \frac{\Delta \vec v}{\Delta t} = \frac{\vec v_f - \vec v_i}{t_f - t_i}$$
\end{defn}
First we note that acceleration is a vector quantity because $\Delta \vec v$ is a vector quantity. Acceleration is experienced any time an object is increasing or decreasing its velocity, \emph{any} change in velocity results in acceleration. For example, you must initially accelerate your vehicle in order for it to reach the desired velocity, similarly you must first \emph{accelerate} your vehicle in order to come to a stop and change your velocity to $(+0\m / \s)$. In this course we will consider ony the average acceleration of a moving body and avoid situations where the acceleration a given body is non-uniform.

\textbf{Remark :} It is common to hear the term \emph{de-accelerate}, however this term is rather redundant because the term acceleration refers to any change in velocity, regardless of weather you would like to increase your velocity or bring yourself to a holt $(\vec v = +0\m / \s)$. \\

\textbf{Remark :} If you are wondering why we are no longer working with $\vec v_{av}$, it is because when we were working with average velocity, we were not concerned with the precise velocity of the moving body at a given point in time but rather the "most common" velocity over a time interval. Average acceleration is concerned with changes in \emph{exact} velocities, we will discuss these differences in a latter subsection.

\begin{ex}
A vehicle on the highway changes his velocity from $\vec v_i = 500 \m / \s $[\tx{East}] to $\vec v_f = 612 \m / \s$ [\tx{West}] in $\Delta t = 2\Min$. Compute his average acceleration,
\end{ex}
\begin{soln}
	$\implies$
\vspace*{4cm}
\end{soln}

\begin{defn}
A \textbf{velocity-time graph} is a plot describing the motion of an object, with velocity on the vertical axis and time on the horizontal axes.
\end{defn}
Similar to the analogy of how a Pos V. Time plot helps us understand velocity better, a Velocity v. Time plot will help us understand acceleration better. Again we mention some basic properties, again we take the reference point to be $(0,0)$. Also we take the positive direction of motion to be above the vertical axes. We now mention a proposition similar to a one we have seen earlier.
\begin{prop}
Given a \textbf{Linear} Velocity v. Time plot of a moving body, the slope $m$ of the plot represents the average acceleration, $\vec a_{av}$, of the body.
\end{prop}
\begin{pf}
We prove the result similar to the method we used in the previous section. Let the slope of the Velocity v. Time graph be $m$, let us compute this slope by using the slope formula, namely,
$$m = \frac{y_2 - y_1}{x_2 - x_1}$$
Since the $y-$coordinates on a Velocity v. Time graph are velocity vectors $\vec v$, and the $x-$coordinates are time points, $t$, we can translate this slope formula to the equivalent,
$$m = \frac{\vec v_2 - \vec v_1}{t_2 - t_1}$$
At this point we are free to choose any two coordinate pairs $(\vec v_1,t_1)$, $(\vec v_2, t_2)$, let us choose $(\vec v_f, t_f)$, $(\vec v_i, t_i)$, the final and initial coordinate pairs of the moving body. This gives,
$$m = \frac{\vec v_f - \vec v_i}{t_f - t_i} = \vec a_{av}$$ 
Since we consider motion in a single direction, the average speed and the average velocity only differ in direction.$\qed$
\end{pf}
\subsection{Types of motion from Velocity v. Time Plots}
Similar to before, we will encounter common types of motion and hence it would be useful to make mention of their plots and what they look like.
\begin{enumerate}[label = (\alph*)]
	\item 
        \begin{tikzpicture}
        \begin{axis}[
            my axis style,
            width=\textwidth,
            height=.5\textwidth,
        ]
        
        \addplot[
            domain=0:5,
            thick,
            purple,
            ->
        ]
        {2};
    
        \fill[
            black
        ];
        
        \end{axis}
        \end{tikzpicture}

		\textbf{\large{Properties of type (a):}}
		\begin{itemize}
			\item The slope of the graph is zero, hence $\vec a_{av} = +0\m / \s^2$.
			\item The object is experiencing \textbf{uniform motion}.
			\item The object is moving [East] relative to the reference point (0,0).
		\end{itemize}


	\item 
	\begin{tikzpicture}
        \begin{axis}[
            my axis style,
            width=\textwidth,
            height=.5\textwidth,
			ylabel = $ $,
        ]
        
        \addplot[
            domain=0:5,
            thick,
            purple,
            ->
        ]
        {-2};
    
        \fill[
            black
        ];
        
        \end{axis}
        \end{tikzpicture}
		
		\textbf{\large{Properties of type (b):}}
		\begin{itemize}
			\item The slope of the graph is zero, hence $\vec a_{av} = +0\m / \s^2$.
			\item The object is experiencing \textbf{uniform motion}.
			\item The object is moving [West] relative to the reference point (0,0).
		\end{itemize}


	\item 
	\begin{tikzpicture}
        \begin{axis}[
            my axis style,
            width=\textwidth,
            height=.5\textwidth,
        ]
        
        \addplot[
            domain=0:5,
            thick,
            purple,
            ->
        ]
        {2*x};

		\addplot[
		domain=0:2,
		thick,
		blue,
		dashdotted,
		]
		{4};

		\draw [dashed, blue, thick] (2,0) -- (2,4);
    
        \fill[
            black
        ];
        
        \end{axis}
        \end{tikzpicture}
		
		\textbf{\large{Properties of type (c):}}
		\begin{itemize}
			\item The slope of the graph is $m = +2$, hence $\vec a_{av} = +2\m / \s^2$.
			\item The object experiencing \textbf{uniform acceleration}.
			\item The object is traveling in the [East] direction.
		\end{itemize}

	\item 
	\begin{tikzpicture}
        \begin{axis}[
            my axis style,
            width=\textwidth,
            height=.5\textwidth,
        ]
        
        \addplot[
            domain=0:5,
            thick,
            purple,
            ->
        ]
        {-2*x + 10};

		\addplot[
		domain=0:3,
		thick,
		blue,
		dashdotted,
		]
		{4};

		
		\draw [dashed, blue, thick] (3,0) -- (3,4);
    
        \fill[
            black
        ];
        
        \end{axis}
        \end{tikzpicture}
		
		\textbf{\large{Properties of type (d):}}
		\begin{itemize}
			\item The slope of the graph is $m = -2$, hence $\vec a_{av} = -2\m / \s^2$.
			\item The object experiencing \textbf{uniform acceleration}.
			\item The object is traveling in the [West] direction.
		\end{itemize}

	\end{enumerate}

\subsection{Instantaneous and Average Velocity}
\begin{defn}
\textbf{Uniform acceleration} is motion where the acceleration of the body is fixed.
\end{defn}

\begin{defn}
The \textbf{instantaneous velocity}, $\vec v$ , of an object is the \emph{exact} velocity of an object at a given time $t$
\end{defn}

	
% \subsection{Non-Linear Position v. Time plots}
% We have analyzed plots of Linear Velocity v. Time plots, we can now ask the question; how would the corresponding Position V. time plots look like for that same body in motion? Well if you have a background in Calculus it would be much easier for me to explain this to you using the idea of \emph{ordered derivatives}, however since the reader of these notes is assumed to have at most a sophomore (or Junior) background in mathematics, I will explain it to the best of my ability.\\


% The proposition from the previous section states that for any given any Pos v. Time plot, its slope represents the average velocity of the body in motion at any given time interval, granted that this Pos v. Time plot was \emph{linear}. However in the case where the pos v. time plot is \emph{non-linear}, the slope of the graph is no longer the same everywhere and hence the average velocity varies for different time intervals. 


	


\end{document}
