% Prepared by Calvin Kent
%
% Assignment Template v19.02
%
%%% 20xx0x/MATHxxx/Crowdmark/Ax
%
\documentclass[12pt]{article} %
\usepackage{CKpreamble}
\usepackage{CKassignment}
\usepackage{tkz-euclide}
\usepackage{physunits}


\begin{document}
    \pagenumbering{arabic}
    % Start of class settings ...
    \renewcommand*{\coursecode}{Physics Homework} % renew course code
    \renewcommand*{\assgnnumber}{2} % renew assignment number
    \renewcommand*{\submdate}{August 14, 2021} % renew the date
    \renewcommand*{\studentfname}{Abdullah} % Student first name
    \renewcommand*{\studentlname}{Zubair} % Student last name
    %\renewcommand*{\studentnum}{SNumber} % Student number

    \renewcommand\qedsymbol{$\blacksquare$}
    \setfigpath
    % End of class settings 
    \pagestyle{crowdmark}
    \newgeometry{left=18mm, right=18mm, top=22mm, bottom=22mm} % page is set to default values
    \fancyhfoffset[L,O]{0pt} % header orientation fixed
    % End of class settings
    %%% Note to user:
    % CTRL + F <CHANGE ME:> (without the angular brackets) in CKpreamble to specify graphics paths accordingly.
    % The command \circled[]{} accepts one optional and one mandatory argument.
    % Optional argument is for the size of the circle and mandatory argument is for its contents.
    % \circled{A} produces circled A, with size drawn for letter A. \circled[TT]{A} produces circled A with size drawn for TT.
    % https://github.com/CalvinKent/My-LaTeX
    %%%
    % Crowdmark assignment start
\begin{qstn}[1] % qnumber, qname, qpoints
 Convert the following quantities to $\frac{\m}{\s}$ (Remember to refer to the conversion table)
 \begin{enumerate}
\item $6004 \frac{\ft}{\h}$
 \vspace*{3cm}
 \item $312300 \frac{\cm}{\h}$
 \vspace*{3cm}
 \item $5 \frac{\km}{\h}$
 \vspace*{3cm}
 \item $10^3 \frac{\mi}{\h}$
 \vspace*{3cm}
 \item $566 \frac{\inch}{\Min}$
 \vspace*{3cm}
 \end{enumerate}
 \end{qstn}


 \begin{qstn}[2]
At the University of Waterloo, students may begin to feel nervous if during an exam, someone manages to complete it after $5$ minutes. Lets say the fastest problem solver in the exam room is student $X$, who solves problems at a rate of $120 \frac{\text{problems}}{\h}$. Determine weather or not the students in the exam room will feel nervous or not, completely justify your answer.
\end{qstn}


\begin{qstn}[3]
Daniel has recently ran into a potentially lucrative opportunity, he happened to come across $60$ carrots of gold. He wants to know how many coffees he can order. He knows the following information, 
\begin{itemize}
\item $1$ carrot of gold = $0.5$ brits
\item $1$ brit = $6000$ USD
\item $1$ USD = $1.25$ CAD
\item $1$ coffee = $2$ CAD
\end{itemize}
Help him determine the number of coffees he can order.

\end{qstn}


\begin{qstn}[4]
    \textbf{(CHALLENGE WARNING)} A mechanical engineering student over at the University of Waterloo wants to
    know the amount of energy he will need in order to weld a $6$ rods of steel. He knows that each rod of
    steel has a density of $650 \kg / \m^3$ and a Calorific Value of $6 \text{ kWh} / \kg$. Determine the
    amount of energy \textbf{(In Kila Jouls)} $6$ rods of steel will require. Make note of the following,
    \begin{itemize}
        \item $1$ rod of steel has a volume of $100 \ft^3$
        \item $1$ BTU $= 2.931 \times 10^{-4} \text{kWh}$
        \item $1 \J = 9.4782 \times 10^{-4} \text{ BTU}$
        \item $1 \text{ k}\J = 1000 \J$
    \end{itemize}

\end{qstn}

\end{document}
%   \begin{figure}[H]
%   \centering
%   \includegraphics[width=0.75\linewidth]{p}
%   \caption{caption.\label{fig:}}
%   \end{figure}
