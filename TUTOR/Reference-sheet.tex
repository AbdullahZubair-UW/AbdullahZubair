\documentclass[12pt]{article} 

\usepackage{fullpage}
\usepackage{bookmark}
\usepackage{amsmath}
\usepackage{amssymb}
\usepackage[dvipsnames]{xcolor}
\usepackage{hyperref} % for the URL
\usepackage[shortlabels]{enumitem}
\usepackage{mathtools}
\usepackage[most]{tcolorbox}
\usepackage[amsmath,standard,thmmarks]{ntheorem} 
\usepackage{physics}
\usepackage{pst-tree} % for the trees
\usepackage{verbatim} % for comments, for version control
\usepackage{tabu}
\usepackage{tikz}
\usepackage{float}
\usepackage{siunitx}
\usepackage{physunits}
\usepackage{pgfplots}

% From the Plot video

\usepackage[LGR,T1]{fontenc}
\usepackage[utf8]{inputenc}
\usepackage{lmodern}
\usepackage{microtype}
\usepackage{upgreek}
\usepackage[misc]{ifsym}

\usepackage{pgfplots}
	\usetikzlibrary{
		calc,
		patterns,
		positioning
	}
	\pgfplotsset{
		compat=1.16,
		samples=200,
		clip=false,
		my axis style/.style={
			axis x line=middle,
			axis y line=middle,
			legend pos=outer north east,
			axis line style={
				->,
			},
			legend style={
				font=\footnotesize
			},
			label style={
				font=\footnotesize
			},
			tick label style={
				font=\footnotesize
			},
			xlabel style={
				at={
					(ticklabel* cs:1)
				},
				anchor=west,
				font=\footnotesize,
			},
			ylabel style={
				at={
					(ticklabel* cs:1)
				},
				anchor=west,
				font=\footnotesize,
			},
			xlabel=$t$,
			ylabel=$\vec v (\m \tx{[East]})$
		},
	}
	\tikzset{
		>=stealth
	}


    \pgfplotsset{my style/.append style={axis x line=middle, axis y line=
           middle, xlabel={$t$}, axis equal }}

%%% Tables and figures packages

\usepackage{float}
\usepackage{caption}
	\captionsetup{
		format=plain,
		labelfont=bf,
		font=small,
		justification=centering
	}
	
%%% Numbers and sets

\newcommand{\E}{\mathrm{e}}
\newcommand{\tx}[1]{\text{#1}}


% floor, ceiling, set
\DeclarePairedDelimiter{\ceil}{\lceil}{\rceil}
\DeclarePairedDelimiter{\floor}{\lfloor}{\rfloor}
\DeclarePairedDelimiter{\set}{\lbrace}{\rbrace}
\DeclarePairedDelimiter{\iprod}{\langle}{\rangle}

\DeclareMathOperator{\Int}{int}
\DeclareMathOperator{\mean}{mean}

% commonly used sets
\newcommand{\R}{\mathbb{R}}
\newcommand{\Nat}{\mathbb{N}}
\newcommand{\Q}{\mathbb{Q}}
\renewcommand{\P}{\mathbb{P}}

\newcommand{\sset}{\subseteq}


\theoremstyle{break}
\theorembodyfont{\upshape}

\newtheorem*{thm}{Theorem}[subsection]
\tcolorboxenvironment{thm}{
enhanced jigsaw,
colframe=Dandelion,
colback=White!90!Dandelion,
drop fuzzy shadow east,
rightrule=2mm,
sharp corners,
before skip=10pt,after skip=10pt
}

\newtheorem*{cor}{Corollary}[thm]
\tcolorboxenvironment{cor}{
boxrule=0pt,
boxsep=0pt,
colback={White!90!RoyalPurple},
enhanced jigsaw,
borderline west={2pt}{0pt}{RoyalPurple},
sharp corners,
before skip=10pt,
after skip=10pt,
breakable
}

\newtheorem{algo}[thm]{Algorithm}
\tcolorboxenvironment{algo}{
enhanced jigsaw,
colframe=Red,
colback={White!95!Red},
rightrule=2mm,
sharp corners,
before skip=10pt,after skip=10pt
}

\newtheorem{ex}[thm]{Example}
\tcolorboxenvironment{ex}{% from ntheorem
blanker,left=5mm,
sharp corners,
before skip=10pt,after skip=10pt,
borderline west={2pt}{0pt}{Green}
}

\newtheorem*{pf}{Proof}
\tcolorboxenvironment{pf}{% from ntheorem
breakable,blanker,left=5mm,
sharp corners,
before skip=10pt,after skip=10pt,
borderline west={2pt}{0pt}{NavyBlue!80!white}
}


\newtheorem*{soln}{Solution}
\tcolorboxenvironment{soln}{% from ntheorem
breakable,blanker,left=5mm,
sharp corners,
before skip=10pt,after skip=10pt,
borderline west={2pt}{0pt}{NavyBlue!80!white}
}

\newtheorem*{defn}{Definition}[subsection]
\tcolorboxenvironment{defn}{
enhanced jigsaw,
colframe=Cerulean,
colback=White!90!Cerulean,
drop fuzzy shadow east,
rightrule=2mm,
sharp corners,
before skip=10pt,after skip=10pt
}

\newtheorem*{prop}[thm]{Proposition}
\tcolorboxenvironment{prop}{
boxrule=0pt,
boxsep=0pt,
colback={White!90!Green},
enhanced jigsaw,
borderline west={2pt}{0pt}{Green},
sharp corners,
before skip=10pt,
after skip=10pt,
breakable
}

\setlength\parindent{0pt}
\setlength{\parskip}{2pt}


\begin{document}
\let\ref\Cref

\begin{center}
	\LARGE{\textbf{Reference Sheet for Test 1}}
\end{center}

%%%%%%%%%%%%%%%%%%%%%%%%%%%%%%%%%%%%%%%%%%%%%%%%%%%%%%%%%%%%%%%%%%%%%%%%%%%%%%%%%%%%%%%%%%%%%%%%%%%%%%%%%%%%%%%%%%%%%%%%%%%%%%%%%%%%%%%%%%%%%%%%
%%%%%%%%%%%%%%%%%%%%%%%%%%%%%%%%%%%%%%%%%%%%%%%%%%%%%%%%%%%%%%%%%%%%%%%%%%%%%%%%%%%%%%%%%%%%%%%%%%%%%%%%%%%%%%%%%%%%%%%%%%%%%%%%%%%%%%%%%%%%%%%%
%%%%%%%%%%%%%%%%%%%%%%%%%%%%%%%%%%%%%%%%%%%%%%%%%%%%%%%%%%%%%%%%%%%%%%%%%%%%%%%%%%%%%%%%%%%%%%%%%%%%%%%%%%%%%%%%%%%%%%%%%%%%%%%%%%%%%%%%%%%%%%%%

\section{S2:}

\begin{defn}
	\textbf{Kinematics} is the study of the motion of an object.
\end{defn}

\begin{defn}
The \textbf{distance} traveled by an object is the total length of the path it traversed.
\end{defn}


\begin{defn}
The \textbf{direction} of an object is an indication of its location according to some system.
\end{defn}

\begin{defn}
	A \textbf{scalar} is a quantity that has a magnitude (size) \emph{only}.
	\end{defn}

\begin{defn}
	A \textbf{vector} is a quantity that has a magnitude as well an associated direction. We denote a variable who represents a vector with an arrow, $\rightarrow$.
\end{defn}

\begin{defn}
	\textbf{Position}, $\vec d$, is the distance and direction of an object from a particular \emph{reference point}.
\end{defn}

\begin{defn}
    The \textbf{displacement} of an object is the change in its position. $$\Delta \vec d = \vec d_{f} - \vec d_{i}$$
\end{defn}

\begin{cor}
    If an object changes its position more than once, the \emph{net} or \emph{total}  displacement of the
    object is computed as, $$\Delta \vec d_T = \vec d_F - \vec d_I$$
    Where $\vec d_F$, $\vec d_I$ are the final and initial position vectors respectively.
\end{cor}


\begin{thm}
	The \textbf{distance} covered by an object is the sum of all displacements along its path
	$$ d = \sum_i |\overrightarrow{\Delta d_i}|$$
\end{thm}

\begin{defn}
		To add two vectors, $\vec A + \vec B$, \emph{Geometrically}, we place the \emph{tail} of $\vec B$ to the
		\emph{tip} of $\vec A$.
\end{defn}	

\begin{defn}
    For a vector $\vec A$, we define $\overrightarrow {-A}$ to be the vector such that $\vec A +
    (\overrightarrow{-A}) = \vec 0$.
\end{defn}

\begin{prop}
	For reference points $A,B$,
$$-\vec d_{AB} = \vec d_{BA}$$
\end{prop}

\begin{prop}
	For reference points $A,B,C$, $$\vec d_{AC} = \vec d_{AB} + \vec d_{BA}$$ 
 \end{prop}



%%%%%%%%%%%%%%%%%%%%%%%%%%%%%%%%%%%%%%%%%%%%%%%%%%%%%%%%%%%%%%%%%%%%%%%%%%%%%%%%%%%%%%%%%%%%%%%%%%%%%%%%%%%%%%%%%%%%%%%%%%%%%%%%%%%%%%%%%%%%%%%%%%%%%%%%%%%%%%%%%%%%%%%%%%%%%%%%%%%%%%%%%%%%%%%%%%%%%%%%%%%%%%%%%%%%%%%%%%%%%%%%%%%%%%%%%%%%%%%%%%%%%%%%%%%%%%%%%%%%%%%%%%%%%%%%%%%%%%%%%%%%%%%%%%%%%%%%%%%%%%%%%%%%%%%%%%%%%%%%%%%%%%%%%%%%%%%%%%%%%%%%%%%%%%%%%%%%%%%%%%%%%%%%%%%%%%%%%%%%%%%%%%%%%%%%%%%%%%%%%%%%%%%%%%%%%%%%%%%%%%%%%%%%%%%%%%%%%%%%%%%%%%%%%%


\section{S3:}

\begin{defn}
The \textbf{average speed} of an object is the ratio of the total distance traveled to the time elapsed. 
$$v_{av} = \frac{d}{\Delta t}$$
Here $\Delta t = t_f - t_i$, or in other words the elapsed time.
\end{defn}


\begin{defn}
    The \textbf{average velocity} of an object is the ratio of the displacement to the time elapsed
    $$\vec v_{av} = \frac{\overrightarrow{\Delta d}}{\Delta t}$$
\end{defn}

\begin{defn}
    A \textbf{position-time graph} is a graph of position versus time. We plot a set of position vectors on the vertical axes with their corresponding time on the horizontal axes. 
\end{defn}

\begin{defn}
	\textbf{Uniform motion} (Or constant velocity) is motion where the velocity is fixed.
\end{defn}

\begin{defn}
	\textbf{Non-uniform motion} (Or accelerated motion) is motion where the velocity is \emph{not} fixed.
\end{defn}

\begin{prop}
	Given a $\textbf{Linear}$ Position v. Time plot of a moving body, the slope $m$ represents the average velocity of the body.
\end{prop}

%%%%%%%%%%%%%%%%%%%%%%%%%%%%%%%%%%%%%%%%%%%%%%%%%%%%%%%%%%%%%%%%%%%%%%%%%%%%%%%%%%%%%%%%%%%%%%%%%%%%%%%%%%%%%%%%%%%%%%%%%%%%%%%%%%%%%%%%%%%%%%%%%%%%%%%%%%%%%%%%%%%%%%%%%%%%%%%%%%%%%%%%%%%%%%%%%%%%%%%%%%%%%%%%%%%%%%%%%%%%%%%%%%%%%%%%%%%%%%%%%%%%%%%%%%%%%%%%%%%%%%%%%%%%%%%%%%%%%%%%%%%%%%%%%%%%%%%%%%%%%%%%%%%%%%%%%%%%%%%%%%%%%%%%%%%%%%%%%%%%%%%%%%%%%%%%%%%%%%%%%%%%%%%%%%%%%%%%%%%%%%%%%%%%%%%%%%%%%%%%%%%%%%%%%%%%%%%%%%%%%%%%%%%%%%%%%%%%%%%%





	


\end{document}
