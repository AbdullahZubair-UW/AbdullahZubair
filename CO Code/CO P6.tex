\documentclass[11pt]{book}
\usepackage[utf8]{inputenc} % Para caracteres en español
\usepackage{amsmath,amsthm,amsfonts,amssymb,amscd}
\usepackage{multirow,booktabs}
\usepackage[table]{xcolor}
\usepackage{fullpage}
\usepackage{tikz}
\usepackage{physics}

%%
\usepackage{pgfplots}
\usepgfplotslibrary{polar}
\usepgflibrary{shapes.geometric}
\usetikzlibrary{calc}


\pgfplotsset{my style/.append style={axis x line=middle, axis y line=
           middle, xlabel={$x$}, ylabel={$y$}, axis equal }}
\pgfplotsset{my style-2/.append style={axis x line=middle, axis y line=
           middle, xlabel={$x$}, ylabel={$y$}}}


\usetikzlibrary{arrows, circuits.ee.IEC, positioning}
\usepackage[american voltages, american currents,siunitx]{circuitikz}

\usepackage{lastpage}
\usepackage{enumitem}
\usepackage{fancyhdr}
\usepackage{mathrsfs}
\usepackage{wrapfig}
\usepackage{setspace}
\usepackage{calc}
\usepackage{multicol}
\usepackage{cancel}
\usepackage[retainorgcmds]{IEEEtrantools}
\usepackage[margin=3cm]{geometry}
\usepackage{amsmath}
\newlength{\tabcont}
\setlength{\parindent}{0.0in}
\setlength{\parskip}{0.05in}
\usepackage{empheq}
\usepackage{framed}
\usepackage[most]{tcolorbox}
\usepackage{xcolor}
\colorlet{shadecolor}{orange!15}
\parindent 0in
\parskip 12pt
\geometry{margin=1in, headsep=0.25in}
\theoremstyle{definition}
\newtheorem{defn}{Definition}
\newtheorem{reg}{Rule}
\newtheorem{exer}{Exercise}
\newtheorem{note}{Note}
\usepackage[english]{babel}
\usepackage[protrusion=true,expansion=true]{microtype}  
\usepackage{amsmath,amsfonts,amsthm,amssymb}
\usepackage{graphicx}
\usepackage{tcolorbox}
\usepackage{amssymb}
\tcbuselibrary{theorems}
\newtcbtheorem
  []% init options
  {definition}% name
  {Definition}% title
  {%
    colback=green!5,
    colframe=green!35!black,
    fonttitle=\bfseries,
  }% options
  {def}% prefix

\newtcbtheorem
  []% init options
  {lemma}% name
  {Lemma}% title
  {%
    colback=yellow!5,
    colframe=orange!110,
    fonttitle=\bfseries,
  }% options
  {lem}% prefix

\newtcbtheorem
  []% init options
  {theorem}% name
  {Theorem}% title
  {%
    colback=yellow!5,
    colframe=orange!110,
    fonttitle=\bfseries,
  }% options
  {thm}% prefix

\newtcbtheorem
  []% init options
  {solution}% name
  {Solution}% title
  {%
    colback=magenta!5,
    colframe=magenta!150,
    fonttitle=\bfseries,
  }% options
  {not}% prefix

\newtcbtheorem
  []% init options
  {example}% name
  {Example}% title
  {%
    colback=red!5,
    colframe=red!100,
    fonttitle=\bfseries,
  }% options
  {exm}% prefix


\newtcbtheorem
  []% init options
  {algorithm}% name
  {Algorithm} %title
  {%
    colback=blue!5,
    colframe=blue!100,
    fonttitle=\bfseries,
  }% options
  {alg}% prefix

\newtcbtheorem
  []% init options
  {properties}% name
  {Properties}% title
  {%
    colback=green!5,
    colframe=green!35!black,
    fonttitle=\bfseries,
  }% options
  {pro}% prefix



% Sets
\newcommand{\N}{\ensuremath{\mathbb{N}}}
\newcommand{\Z}{\ensuremath{\mathbb{Z}}}
\newcommand{\Q}{\ensuremath{\mathbb{Q}}}
\newcommand{\R}{\ensuremath{\mathbb{R}}}
\newcommand{\C}{\ensuremath{\mathbb{C}}}
\newcommand{\F}{\ensuremath{\mathbb{F}}}
\newcommand{\Mnn}{\ensuremath{M_{n \times n}}}
\newcommand{\sym}{\mathbin{\triangle}}
\newcommand{\stcomp}[1]{{#1}^\complement}

% Operators
\newcommand{\Rarr}{\Rightarrow}
\newcommand{\Larr}{\Leftarrow}
\newcommand{\Harr}{\Leftrightarrow}
\newcommand{\harr}{\leftrightarrow}
\newcommand{\dyx}{\dv{y}{x}}

% Linear Algebra
\newcommand{\vui}{\mathbf{\hat{\textnormal{\bfseries\i}}}}
\newcommand{\vuj}{\mathbf{\hat{\textnormal{\bfseries\j}}}}
\newcommand{\xto}{\xrightarrow} % \xto{R_1 \harr R_2}
\newcommand{\cbm}[2]{\ensuremath{{}_{#1}[I]{}_{#2}}} % change of basis matrix
\DeclareMathOperator{\Proj}{Proj}
\DeclareMathOperator{\Perp}{Perp}
\DeclareMathOperator{\Span}{Span}
\DeclareMathOperator{\Col}{Col}
\DeclareMathOperator{\adj}{adj}

\newcommand{\QType}{Q}
\newcounter{question}[subsection]
\renewcommand{\thequestion}{\QType\ifnum\value{question}<10 0\fi\arabic{question}}
\newcommand{\question}{\par\refstepcounter{question}\textbf{\thequestion}.\space}

\renewcommand\qedsymbol{$\blacksquare$}

\begin{document}

(A-5-4 MATH 239, Winter 2021) Let $G$ be a graph such that the edge set does not cotain three distinct vertices $x,y,z$ such that $xy, yz \in E(G)$ and $xz\not\in E(G)$. Prove that there exists $k\in \N$ and sets $V_1,\dots,V_k$ such that:
\begin{itemize}
\item $V_1\cup \cdots \cup V_k = V(G)$
\item $V_i \cap V_j = \emptyset$ for all $i \neq j$
\item For all $i$, we have $xy\in E(G)$ for all distinct $x,y\in V_i$
\item For all $i\neq j$, we have $xy\in E(G)$ for all $x\in V_i$ and $y\in V_j$.
\end{itemize}

\textbf{\underline{Solution:}} Note that for any given subgraph $G'$, $|V(G')| = 3$, the corrosponding edge set will have size $|E(G)| = 3,1,0$. If $|E(G)| = 2$, then for some $i,j,k$ $\{v_i,v_j\}, \{v_j,v_k\} \in E(G')$ and $v_iv_k\not\in E(G')$, which violates properties of $G$. We construct sets $V_1,\dots,V_k$ using the following algorithm, let $|V(G)| = n$, for each vertex $v_i\in V(G)$, let $U = N(v_i) \cup v_i$, then preform $V_j = V_j \cup U$ for some set $V_j$. Repeat the algorithm for the next vertex $v_r$, $r\geq i$, such that $v_r\not N(v_i)$ and for the next set $V_{j + 1}$. This will give us $k-subgraphs$, $V_1,\dots,v_k$, $k\leq n$.

Clearly $V_1\cup \cdots \cup V_k = V(G)$ since we consider all vertices $v_i \in V(G)$ and their corrosponding neighbours. Also, note that $V_i\cap V_j = \emptyset$ for all $i\neq j$, if we let $V_i,V_j$ repersent the neighbourhood of $v_i,v_j$ and if there existed some $v\in V(G)$ such that $v\in V_j$ and $v\in V_i$, $i\neq j$, then the subgraph $G'$ conataing vertices $v_i,v_j,v$ would contain an edge set $v_iv, vv_j$, however $v_iv_j\not\in E(G')$, a violation. Observe that each distinct $x,y\in V_i$ implies that $xy\in E(G)$, this follows from the fact that $V_i$ is the neighbourhood of some vertex $v$, hence $xv, vy\in E(G)$, however from properties of $G$, if $xv,vy\in E(G)$ then $xy\in E(G)$. Also for each $i\neq j$, if $x\in V_i$ and $y\in V_j$, then $xy\not\in E(G)$. This follows from the previous argument, $V_i$ repersents the neighbourhood of some vertex $v$, if $x = v$, then clearly $xy\not\in E(G)$, since $V_i\neq V_j$. Else, if we assume that $xy \in E(G)$, the the triplet $x,y,v$ would produce an edge set $vx, xy\in E(G)$ and $vy\not\in E(G)$, a violation of properites of $G$. \qed
\end{document}

