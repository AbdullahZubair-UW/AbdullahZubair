\documentclass[11pt]{book}
\usepackage[utf8]{inputenc} % Para caracteres en español
\usepackage{amsmath,amsthm,amsfonts,amssymb,amscd}
\usepackage{multirow,booktabs}
\usepackage[table]{xcolor}
\usepackage{fullpage}
\usepackage{tikz}
\usepackage{physics}

%%
\usepackage{pgfplots}
\usepgfplotslibrary{polar}
\usepgflibrary{shapes.geometric}
\usetikzlibrary{calc}


\pgfplotsset{my style/.append style={axis x line=middle, axis y line=
           middle, xlabel={$x$}, ylabel={$y$}, axis equal }}
\pgfplotsset{my style-2/.append style={axis x line=middle, axis y line=
           middle, xlabel={$x$}, ylabel={$y$}}}


\usetikzlibrary{arrows, circuits.ee.IEC, positioning}
\usepackage[american voltages, american currents,siunitx]{circuitikz}

\usepackage{lastpage}
\usepackage{enumitem}
\usepackage{fancyhdr}
\usepackage{mathrsfs}
\usepackage{wrapfig}
\usepackage{setspace}
\usepackage{calc}
\usepackage{multicol}
\usepackage{cancel}
\usepackage[retainorgcmds]{IEEEtrantools}
\usepackage[margin=3cm]{geometry}
\usepackage{amsmath}
\newlength{\tabcont}
\setlength{\parindent}{0.0in}
\setlength{\parskip}{0.05in}
\usepackage{empheq}
\usepackage{framed}
\usepackage[most]{tcolorbox}
\usepackage{xcolor}
\colorlet{shadecolor}{orange!15}
\parindent 0in
\parskip 12pt
\geometry{margin=1in, headsep=0.25in}
\theoremstyle{definition}
\newtheorem{defn}{Definition}
\newtheorem{reg}{Rule}
\newtheorem{exer}{Exercise}
\newtheorem{note}{Note}
\usepackage[english]{babel}
\usepackage[protrusion=true,expansion=true]{microtype}  
\usepackage{amsmath,amsfonts,amsthm,amssymb}
\usepackage{graphicx}
\usepackage{tcolorbox}
\usepackage{amssymb}
\tcbuselibrary{theorems}
\newtcbtheorem
  []% init options
  {definition}% name
  {Definition}% title
  {%
    colback=green!5,
    colframe=green!35!black,
    fonttitle=\bfseries,
  }% options
  {def}% prefix

\newtcbtheorem
  []% init options
  {lemma}% name
  {Lemma}% title
  {%
    colback=yellow!5,
    colframe=orange!110,
    fonttitle=\bfseries,
  }% options
  {lem}% prefix

\newtcbtheorem
  []% init options
  {theorem}% name
  {Theorem}% title
  {%
    colback=yellow!5,
    colframe=orange!110,
    fonttitle=\bfseries,
  }% options
  {thm}% prefix

\newtcbtheorem
  []% init options
  {solution}% name
  {Solution}% title
  {%
    colback=magenta!5,
    colframe=magenta!150,
    fonttitle=\bfseries,
  }% options
  {not}% prefix

\newtcbtheorem
  []% init options
  {example}% name
  {Example}% title
  {%
    colback=red!5,
    colframe=red!100,
    fonttitle=\bfseries,
  }% options
  {exm}% prefix


\newtcbtheorem
  []% init options
  {algorithm}% name
  {Algorithm} %title
  {%
    colback=blue!5,
    colframe=blue!100,
    fonttitle=\bfseries,
  }% options
  {alg}% prefix

\newtcbtheorem
  []% init options
  {properties}% name
  {Properties}% title
  {%
    colback=green!5,
    colframe=green!35!black,
    fonttitle=\bfseries,
  }% options
  {pro}% prefix



% Sets
\newcommand{\N}{\ensuremath{\mathbb{N}}}
\newcommand{\Z}{\ensuremath{\mathbb{Z}}}
\newcommand{\Q}{\ensuremath{\mathbb{Q}}}
\newcommand{\R}{\ensuremath{\mathbb{R}}}
\newcommand{\C}{\ensuremath{\mathbb{C}}}
\newcommand{\F}{\ensuremath{\mathbb{F}}}
\newcommand{\Mnn}{\ensuremath{M_{n \times n}}}
\newcommand{\sym}{\mathbin{\triangle}}
\newcommand{\stcomp}[1]{{#1}^\complement}

% Operators
\newcommand{\Rarr}{\Rightarrow}
\newcommand{\Larr}{\Leftarrow}
\newcommand{\Harr}{\Leftrightarrow}
\newcommand{\harr}{\leftrightarrow}
\newcommand{\dyx}{\dv{y}{x}}

% Linear Algebra
\newcommand{\vui}{\mathbf{\hat{\textnormal{\bfseries\i}}}}
\newcommand{\vuj}{\mathbf{\hat{\textnormal{\bfseries\j}}}}
\newcommand{\xto}{\xrightarrow} % \xto{R_1 \harr R_2}
\newcommand{\cbm}[2]{\ensuremath{{}_{#1}[I]{}_{#2}}} % change of basis matrix
\DeclareMathOperator{\Proj}{Proj}
\DeclareMathOperator{\Perp}{Perp}
\DeclareMathOperator{\Span}{Span}
\DeclareMathOperator{\Col}{Col}
\DeclareMathOperator{\adj}{adj}

\newcommand{\QType}{Q}
\newcounter{question}[subsection]
\renewcommand{\thequestion}{\QType\ifnum\value{question}<10 0\fi\arabic{question}}
\newcommand{\question}{\par\refstepcounter{question}\textbf{\thequestion}.\space}

\renewcommand\qedsymbol{$\blacksquare$}

\begin{document}

(A-7-3 MATH 239, Winter 2021) Let $G$ be a graph with at least three vertices such that for all $v\in V(G)$, the graph arising from $G$ by deleting $v$ is a tree. Prove that $G$ is a cycle.

\textbf{\underline{Solution:}}  Let $G = (V,E)$ denote the graph, since $|V(G)| \geq 3$, $G$ is not-empty. Let $v\in V(G)$, we denote $T$ as the tree arising from $G$ by deleting $v$. Since $T$ is a tree, it is minimally connected and hence has property $c(T) = 1$. Assume, for the purposes of a contradiction, that $e = vw \in E(G)$ is a bridge of $G$ where $w\in V(G)$ denotes the neighbour of $v$, it follows that $c(G \setminus e) > c(G)$ by definition of a bridge. However since $c(T) = 1$ and $c(G) \neq 0$, the deletion of all $vw_i \in E(G)$, where each $w_i \in V(G)$ denotes the neighbour of $v$, has no effect on the connectivity of $G$, and hence $c(G) = 1$. This would also imply that each edge of $G$ is not a bridge and is hence contained in a cycle.

We now prove that $G$ is $2$-regular. Clearly there cannot exists a $v\in V(G)$ such that $\deg (v) <  2$, else this would imply the existence of a bridge, contradicting our proof above. Suppose that there exists a vertex $v_0\in V(G)$ such that $\deg (v_0) \geq 2$. We denote $C_1 = (v_0\cdots v_jv_0)$ to be the cycle that $v_0$ is contained within. Since $v_0$ has some neighbour outside of $C_1$, there must exist some distinct cycle $C_2 = (v_0 \cdots v_kv_0)$ in which $v_0$ is also contained within, this follows from our proof above, that being that all edges must be contained within a cycle. Since these cycles are distinct, there is some $w\in C_1, w\not\in C_2$, such that the deletion of $w$ will preserve adjacency on $C_1$. However since $C_1$ is a cycle, a tree $T$ does not arise. Hence $G$ is a connected $2$-regular graph, and is hence a cycle. 
\qed
\end{document}

