% Prepared by Calvin Kent
%
% Lecture Template v19.02
%
%%% 201901/MATHxxx/Notes
%
\documentclass[12pt,oneside]{book} %
\usepackage{CKpreamble}
\usepackage{CKlecture}
%
\renewcommand*{\doctitle}{Chapter Based Lecture Notes}
%\makeatletter\patchcmd{\chapter}{\if@openright\cleardoublepage\else\clearpage\fi}{}{}{}\makeatother % only used in class based
\begin{document}
	% Start of Class settings
	\renewcommand*{\term}{Term 2000} % Term
	\renewcommand*{\coursecode}{MATH 000} % Course code
	\renewcommand*{\coursename}{Course Name} % Full course name
	\renewcommand*{\profname}{Prof Name} % Prof Name
	\renewcommand*{\colink}{http://www.student.math.uwaterloo.ca/~c2kent} % Course outline link
	% End of Class settings
	\setfigpath
	\pagenumbering{roman}
	\pagestyle{plain}
	\clearpage
	\pagenumbering{arabic}
	\pagestyle{chapterlecture}
	%%% Note to user: CTRL + F <CHANGE ME:> (without the angular brackets) in CKpreamble to specify graphics paths accordingly.
	%%% If a new chapter was started in the middle of a lecture, %\fixchap{Second Chapter} must be used immediately above the next lecture.
	% Course notes start
	\setchap{2}{Formulations}[1][x]
	% Start of lecture % {00 January 2019}
		\chapter{\chapname\chaplec}
		\section{Introductions}
        In this course we will consider the following problems;
        $$\text{min}\{f(x) \colon g_i(x) \leq b_i, (1 \leq i \leq m), x\in \R^n\}$$
        Whereby,
        \begin{itemize}
            \item $n,m \in \N$
            \item $b_1,\dots,b_m\in\R$
            \item $f,g_1,\dots,d_m$ are functions.
        \end{itemize}
        \begin{defn*}
            A function $f \colon \R^n \rightarrow R$ is \textbf{affine} if $f(x) = a^Tx + \beta$ for $a\in \mathbb R^n$, $b\in \R$. We say that it is \textbf{liner} if $\beta = 0$.
        \end{defn*}
        \begin{defn*}
            The optimization problem
        $$\text{min}\{f(x) \colon g_i(x) \leq b_i, \forall 1 \leq i \leq m, x\in \R^n\}$$
        is called a \textbf{linear} program if $f$ is \textbf{affine} and $g_1,\dots,d_m$ is \textbf{finite} and number of \textbf{linear} functions. 
        \begin{rem}
            We place the non-negativity constraints in the last row. We write $x \geq \mathbb O$ as a short form that denote that all variables are non-negative. THe following are not allowed,
            \begin{itemize}      
                \item Dividing by variables.
                \item Strict inequalities are not allowed. (Non-equality inequalities)
                \item Must have a finite number of constraints.

            \end{itemize}
         \end{rem}

        \end{defn*}
		




	% End of lecture % 
\end{document}
%\fixchap{Second Chapter}
%	\begin{figure}[H]
%	\centering
%	\includegraphics[width=0.75\linewidth]{p}
%	\caption{caption.\label{fig:}}
%	\end{figure}