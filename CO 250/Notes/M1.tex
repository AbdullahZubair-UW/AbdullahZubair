% Prepared by Calvin Kent
%
% Lecture Template v19.02
%
%%% 201901/MATHxxx/Notes
%
\documentclass[11pt,oneside]{book} %
\usepackage{CKpreamble}
\usepackage{CKlecture}
%
\renewcommand*{\doctitle}{Chapter Based Lecture Notes}
%\makeatletter\patchcmd{\chapter}{\if@openright\cleardoublepage\else\clearpage\fi}{}{}{}\makeatother % only used in class based
\begin{document}
	% Start of Class settings
	\renewcommand*{\term}{Term 2000} % Term
	\renewcommand*{\coursecode}{MATH 000} % Course code
	\renewcommand*{\coursename}{Course Name} % Full course name
	\renewcommand*{\profname}{Prof Name} % Prof Name
	\renewcommand*{\colink}{http://www.student.math.uwaterloo.ca/~c2kent} % Course outline link
	% End of Class settings
	\setfigpath
	\pagenumbering{roman}
	\pagestyle{plain}
	\clearpage
	\pagenumbering{arabic}
	\pagestyle{chapterlecture}
	%%% Note to user: CTRL + F <CHANGE ME:> (without the angular brackets) in CKpreamble to specify graphics paths accordingly.
	%%% If a new chapter was started in the middle of a lecture, %\fixchap{Second Chapter} must be used immediately above the next lecture.
	% Course notes start
	\setchap{1}{Formulations}[1][x]
	% Start of lecture % {00 January 2019}
		\chapter{\chapname\chaplec}
		\section{Introduction}
		\begin{defn*}
			An \emph{Abstract optimization problem (P)} is a problem where we are given,
			\begin{itemize}
				\item \textbf{Goal: }A set $A\subseteq \R^n$ and a function $f \colon A \rightarrow \R$
				\item \textbf{Goal: }Find $x\in A$ that min/max $f$.
			\end{itemize}
		\end{defn*}
		\begin{defn*}
			We define different optimization problems;
			\begin{itemize}
				\item \textbf{Linear Programming (LP): } $A$ is implicitly given by linear constraints, and $f$ is a linear function.
				\item \textbf{Integer Programming (IP): } Similar to LP however now we optimize over integer points in $A$.
				\item \textbf{Nonlinear Programming (NLP): } $A$ is given by non-linear constraints, and $f$ is a non-linear function.
			\end{itemize}
		\end{defn*}


		\section{Components of a Math Model}
		\begin{defn*}
			Components;
			\begin{itemize}
				\item \textbf{Decision Variables: } Capture unknown info.
				\item \textbf{Constraints:} Describe which assignments to variables are \emph{feasible}
				\item \textbf{Objective function: }A function of the variables that we are minimizing/maximizing.
			\end{itemize}

		\end{defn*}
		




	% End of lecture % 
\end{document}
%\fixchap{Second Chapter}
%	\begin{figure}[H]
%	\centering
%	\includegraphics[width=0.75\linewidth]{p}
%	\caption{caption.\label{fig:}}
%	\end{figure}