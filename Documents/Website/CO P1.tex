\documentclass[11pt]{book}
\usepackage[utf8]{inputenc} % Para caracteres en español
\usepackage{amsmath,amsthm,amsfonts,amssymb,amscd}
\usepackage{multirow,booktabs}
\usepackage[table]{xcolor}
\usepackage{fullpage}
\usepackage{tikz}
\usepackage{physics}

%%
\usepackage{pgfplots}
\usepgfplotslibrary{polar}
\usepgflibrary{shapes.geometric}
\usetikzlibrary{calc}


\pgfplotsset{my style/.append style={axis x line=middle, axis y line=
           middle, xlabel={$x$}, ylabel={$y$}, axis equal }}
\pgfplotsset{my style-2/.append style={axis x line=middle, axis y line=
           middle, xlabel={$x$}, ylabel={$y$}}}


\usetikzlibrary{arrows, circuits.ee.IEC, positioning}
\usepackage[american voltages, american currents,siunitx]{circuitikz}

\usepackage{lastpage}
\usepackage{enumitem}
\usepackage{fancyhdr}
\usepackage{mathrsfs}
\usepackage{wrapfig}
\usepackage{setspace}
\usepackage{calc}
\usepackage{multicol}
\usepackage{cancel}
\usepackage[retainorgcmds]{IEEEtrantools}
\usepackage[margin=3cm]{geometry}
\usepackage{amsmath}
\newlength{\tabcont}
\setlength{\parindent}{0.0in}
\setlength{\parskip}{0.05in}
\usepackage{empheq}
\usepackage{framed}
\usepackage[most]{tcolorbox}
\usepackage{xcolor}
\colorlet{shadecolor}{orange!15}
\parindent 0in
\parskip 12pt
\geometry{margin=1in, headsep=0.25in}
\theoremstyle{definition}
\newtheorem{defn}{Definition}
\newtheorem{reg}{Rule}
\newtheorem{exer}{Exercise}
\newtheorem{note}{Note}
\usepackage[english]{babel}
\usepackage[protrusion=true,expansion=true]{microtype}  
\usepackage{amsmath,amsfonts,amsthm,amssymb}
\usepackage{graphicx}
\usepackage{tcolorbox}
\usepackage{amssymb}
\tcbuselibrary{theorems}
\newtcbtheorem
  []% init options
  {definition}% name
  {Definition}% title
  {%
    colback=green!5,
    colframe=green!35!black,
    fonttitle=\bfseries,
  }% options
  {def}% prefix

\newtcbtheorem
  []% init options
  {lemma}% name
  {Lemma}% title
  {%
    colback=yellow!5,
    colframe=orange!110,
    fonttitle=\bfseries,
  }% options
  {lem}% prefix

\newtcbtheorem
  []% init options
  {theorem}% name
  {Theorem}% title
  {%
    colback=yellow!5,
    colframe=orange!110,
    fonttitle=\bfseries,
  }% options
  {thm}% prefix

\newtcbtheorem
  []% init options
  {solution}% name
  {Solution}% title
  {%
    colback=magenta!5,
    colframe=magenta!150,
    fonttitle=\bfseries,
  }% options
  {not}% prefix

\newtcbtheorem
  []% init options
  {example}% name
  {Example}% title
  {%
    colback=red!5,
    colframe=red!100,
    fonttitle=\bfseries,
  }% options
  {exm}% prefix


\newtcbtheorem
  []% init options
  {algorithm}% name
  {Algorithm} %title
  {%
    colback=blue!5,
    colframe=blue!100,
    fonttitle=\bfseries,
  }% options
  {alg}% prefix

\newtcbtheorem
  []% init options
  {properties}% name
  {Properties}% title
  {%
    colback=green!5,
    colframe=green!35!black,
    fonttitle=\bfseries,
  }% options
  {pro}% prefix



% Sets
\newcommand{\N}{\ensuremath{\mathbb{N}}}
\newcommand{\Z}{\ensuremath{\mathbb{Z}}}
\newcommand{\Q}{\ensuremath{\mathbb{Q}}}
\newcommand{\R}{\ensuremath{\mathbb{R}}}
\newcommand{\C}{\ensuremath{\mathbb{C}}}
\newcommand{\F}{\ensuremath{\mathbb{F}}}
\newcommand{\Mnn}{\ensuremath{M_{n \times n}}}
\newcommand{\sym}{\mathbin{\triangle}}
\newcommand{\stcomp}[1]{{#1}^\complement}

% Operators
\newcommand{\Rarr}{\Rightarrow}
\newcommand{\Larr}{\Leftarrow}
\newcommand{\Harr}{\Leftrightarrow}
\newcommand{\harr}{\leftrightarrow}
\newcommand{\dyx}{\dv{y}{x}}

% Linear Algebra
\newcommand{\vui}{\mathbf{\hat{\textnormal{\bfseries\i}}}}
\newcommand{\vuj}{\mathbf{\hat{\textnormal{\bfseries\j}}}}
\newcommand{\xto}{\xrightarrow} % \xto{R_1 \harr R_2}
\newcommand{\cbm}[2]{\ensuremath{{}_{#1}[I]{}_{#2}}} % change of basis matrix
\DeclareMathOperator{\Proj}{Proj}
\DeclareMathOperator{\Perp}{Perp}
\DeclareMathOperator{\Span}{Span}
\DeclareMathOperator{\Col}{Col}
\DeclareMathOperator{\adj}{adj}

\newcommand{\QType}{Q}
\newcounter{question}[subsection]
\renewcommand{\thequestion}{\QType\ifnum\value{question}<10 0\fi\arabic{question}}
\newcommand{\question}{\par\refstepcounter{question}\textbf{\thequestion}.\space}

\renewcommand\qedsymbol{$\blacksquare$}

\begin{document}
A binary string $a_1\cdots a_n$ is a palindrome if it reads equivalently forward as backward, or in other words,
$$a_n\cdots a_1 = a_1\cdots a_n$$
We consider $\epsilon$ to be a palindrome. Let the $S$ be the set of all binary strings that are plaindromes. 
\begin{enumerate}
\item Find a formula for the number of $n$ length binary strings in the set $S$ with proof.
\item Let $\omega(a_1\cdots a_n) = \sum_{i = 1}^n(a_i + 1)$. Let $H$ be the set of strings in $S$ with even length. Determine $\Phi_H(x)$ with repsect to $\omega$

\end{enumerate}

\textbf{\underline{Solution:}}
\begin{enumerate}
\item Let $T$ be the set of all length $\lceil n / 2 \rceil$ binary strings. Let $S_n$ be the set of all length $n$ binary strings. We claim a bijection $T \rightleftharpoons S_n$. Let $f \colon T \rightarrow S_n$ be a function by letting $\alpha = b_1\cdots b_{\lceil n / 2 \rceil} \in T$, be mapped to 

$$f(\alpha) = b_1\cdots b_{\lceil n / 2 \rceil}b_{\lceil n / 2 \rceil - 1}\cdots b_1$$

Clearly $l(f(\alpha)) = n$, also the string is symmetric about $b_{\lceil n / 2 \rceil}$ and hence $f(\alpha) \in S_n$. Let $\gamma = b_1\cdots b_n \in S_n$, the we choose $\beta = b_1\cdots b_{\lceil n / 2 \rceil} \in T$ to construct the map $f(\beta) = \gamma$, hence $f$ is surjective. If $f(\gamma) = f(\beta)$ then $a_1\cdots a_{\lceil n / 2 \rceil} = b_1\cdots b_{\lceil n / 2 \rceil}$, hence each bit is equal which implies $\gamma  = \beta$, hence $f$ is injective. Since $f$ is both sujrective and injective, we conclude that there exists a bijection between $S_n$ and $T$, and hence $|T| = |S_n| = 2^{\lceil n / 2 \rceil}$
\item \begin{align*}
\sum_{\alpha \in H}x^{\omega(\alpha)} &= \sum_{k \geq 0}\sum_{\beta\in T} x^{\left(2\sum_{i = 1}^k (b_i + 1)\right)}\\
&= \sum_{k \geq 0}\sum_{\beta \in T}x^{\left(2(b_1 + 1 + \dots + b_k + 1) \right)}\\
&= \sum_{k \geq 0}\sum_{\beta \in T}x^{2(b_1 + 1) + \dots + 2(b_k + 1)}\\
&= \sum_{k \geq 0}\sum_{b_1 = 0}^1x^{2(b_1 + 1)}\sum_{b_2 = 0}^1x^{2(b_2 + 1)} \cdots  \sum_{b_k = 0}^1x^{2(b_k + 1)}\\
&= \sum_{k \geq 0}\sum_{b_1 = 0}^1x^{2(b_1 + 1)}\sum_{b_2 = 0}^1x^{2(b_2 + 1)} \cdots  \sum_{b_k = 0}^1x^{2(b_k + 1)} \\
&= \sum_{k \geq 0}\left(\sum_{b = 0}^1x^{2(b + 1)} \right)^k\\
&= \sum_{k \geq 0}(x^2 + x^4)^k\\
&= \frac{1}{1 - x^2 - x^4}
\end{align*}

\end{enumerate}
\end{document}
